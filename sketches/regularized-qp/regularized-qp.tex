\documentclass[12pt]{article}

% Packages
\usepackage[margin=5em]{geometry} % 1 cm = 2.84528 em
\usepackage[backend=bibtex]{biblatex}
\bibliography{sources}
% \nocite{*}

\usepackage{lipsum}

% Paragraphs
\setlength{\parindent}{0em}
\setlength{\parskip}{1em}

% Includes
% Includes
\usepackage[ruled,vlined]{algorithm2e}
\usepackage{amsfonts}
\usepackage{amsmath}
\usepackage{amssymb}
\usepackage{amsthm}
\usepackage{comment}
\usepackage{float}
\usepackage{graphicx}
\usepackage[colorlinks]{hyperref}
\usepackage{letltxmacro}
\usepackage{listings}
\usepackage{mathtools}
\usepackage{physics}
% \usepackage{stmaryrd}

% Grouping
\let\parens\undefined
\newcommand{\parens}[1]{{\left(#1\right)}}

\let\bracks\undefined
\newcommand{\bracks}[1]{{\left[#1\right]}}

\let\braces\undefined
\newcommand{\braces}[1]{{\left\{#1\right\}}}

\let\angles\undefined
\newcommand{\angles}[1]{{\left\langle#1\right\rangle}}

\let\ceil\undefined
\newcommand{\ceil}[1]{{\left\lceil#1\right\rceil}}

\let\floor\undefined
\newcommand{\floor}[1]{{\left\lfloor#1\right\rfloor}}

\let\abs\undefined
\newcommand{\abs}[1]{{\left\lvert#1\right\rvert}}

\let\norm\undefined
\newcommand{\norm}[1]{{\left\lVert#1\right\rVert}}

% Fonts in math mode
\let\mrm\undefined
\newcommand{\mrm}[1]{{\mathrm{#1}}}

\let\mbb\undefined
\newcommand{\mbb}[1]{{\mathbb{#1}}}

\let\mbf\undefined
\newcommand{\mbf}[1]{{\mathbf{#1}}}

\let\mcal\undefined
\newcommand{\mcal}[1]{{\mathcal{#1}}}

% Operators on functions
\let\mod\undefined
\DeclareMathOperator{\mod}{mod}

\let\ker\undefined
\DeclareMathOperator{\ker}{ker}

\let\dom\undefined
\DeclareMathOperator{\dom}{dom}

\let\ran\undefined
\DeclareMathOperator{\ran}{ran}

\let\im\undefined
\DeclareMathOperator{\im}{im}

\let\argmin\undefined
\DeclareMathOperator{\argmin}{argmin}

\let\argmax\undefined
\DeclareMathOperator{\argmax}{argmax}

\let\arginf\undefined
\DeclareMathOperator{\arginf}{arginf}

\let\argsup\undefined
\DeclareMathOperator{\argsup}{argsup}

\let\minimize\undefined
\DeclareMathOperator{\minimize}{minimize}

\let\maximize\undefined
\DeclareMathOperator{\maximize}{maximize}

% Sets and topology
\let\intr\undefined
\DeclareMathOperator{\intr}{int}

\let\clos\undefined
\DeclareMathOperator{\clos}{clos}

\let\bndr\undefined
\DeclareMathOperator{\bndr}{bndr}

\let\dist\undefined
\DeclareMathOperator{\dist}{dist}

% Numbers
\let\Re\undefined
\DeclareMathOperator{\Re}{Re}

\let\Im\undefined
\DeclareMathOperator{\Im}{Im}

\let\Arg\undefined
\DeclareMathOperator{\Arg}{Arg}

\let\mod\undefined
\DeclareMathOperator{\mod}{mod}

\let\sign\undefined
\DeclareMathOperator{\sign}{sign}

% Linear algebra and convex analysis
\let\rank\undefined
\DeclareMathOperator{\rank}{rank}

\let\det\undefined
\DeclareMathOperator{\det}{det}

\let\Tr\undefined
\DeclareMathOperator{\Tr}{Tr}

\let\diag\undefined
\DeclareMathOperator{\diag}{diag}

\let\conv\undefined
\DeclareMathOperator{\conv}{conv}

\let\epi\undefined
\DeclareMathOperator{\epi}{epi}

\let\zeros\undefined
\DeclareMathOperator{\zeros}{\mathbf{0}}

\let\ones\undefined
\DeclareMathOperator{\ones}{\mathbf{1}}

% Probability
\let\Pr\undefined
\DeclareMathOperator{\Pr}{Pr}

\let\E\undefined
\DeclareMathOperator{\E}{\mbf{E}}

\let\Var\undefined
\DeclareMathOperator{\Var}{Var}

\let\Cov\undefined
\DeclareMathOperator{\Cov}{Cov}

% Computational Complexity
\let\class\undefined % Complexity class
\newcommand{\class}[1]{{{\textbf{#1}}}}

% Theorem environment
\theoremstyle{plain}
\newtheorem{theorem}{Theorem}

\theoremstyle{plain}
\newtheorem*{theorem*}{Theorem}

\theoremstyle{plain}
\newtheorem{lemma}{Lemma}

\theoremstyle{plain}
\newtheorem*{lemma*}{Lemma}

\theoremstyle{definition}
\newtheorem{definition}{Definition}

\theoremstyle{definition}
\newtheorem*{definition*}{Definition}

\theoremstyle{definition}
\newtheorem{question}{Question}

\theoremstyle{definition}
\newtheorem*{question*}{Question}

\theoremstyle{definition}
\newtheorem{problem}{Problem}

\theoremstyle{definition}
\newtheorem*{problem*}{Problem}

\theoremstyle{definition}
\newtheorem{solution}{Solution}

\theoremstyle{definition}
\newtheorem*{solution*}{Solution}

\theoremstyle{definition}
\newtheorem{assumption}{Assumption}

\theoremstyle{definition}
\newtheorem*{assumption*}{Assumption}

% Code environment
\lstdefinestyle{plainsty}{
  basicstyle=\small\ttfamily,
  language=C,
  xleftmargin=\parindent,
  aboveskip=1em,
  belowskip=1em,
  showspaces=false,
  showstringspaces=false,
  keywordstyle = {},
}

\lstnewenvironment{pcode}{\lstset{style=plainsty}}{}

\let\code\undefined
\newcommand{\code}[1]{\texttt{#1}}


% Abbreviations
\let\st\undefined
\newcommand{\st}{\text{s.t.}}

\let\ow\undefined
\newcommand{\ow}{\text{otherwise}}

\let\eg\undefined
\newcommand{\eg}{\text{e.g.}}

\let\ie\undefined
\newcommand{\ie}{\text{i.e.}}

\let\cf\undefined
\newcommand{\cf}{\text{c.f.}}

% Custom text highlighting
\let\red\undefined
\newcommand{\red}[1]{\textbf{\color{red}#1}}





% Author
\title{Regularized Equality Constrained Quadratic Optimization}
\author{Anton Xue and Nikolai Matni}
\date{\today}
\date{}

% Document
\begin{document}
\maketitle

%%%%%%%%%%%%%%%%%%%%%%%%%%%%%%%%%%%%%%%%%%%%%%%%%%%%%%%%%%%%
\section{Introduction}
We look at regularized approximations
of equality-constrained quadratic programming.
In particular, what is \(\norm{x^\star - x_{\lambda} ^\star}\),
where \(x^\star\) is the optimal solution to the original problem,
while \(x_{\lambda} ^\star\) is the solution to the regularized problem.


\section{Background}
Stephen Boyd and Lieven Vandenberghe~\cite{boyd2004convex}.

\subsection{Equality Constrained Quadratic Programming}

A quadratic program with equality constraints~\cite{boyd2004convex} is
\begin{align}
  \text{minimize} &\quad \frac{1}{2} x^\top Q x
    \label{eqn:qp} \\
  \text{subject to} &\quad Ax = b
\end{align}
with variable in \(x \in \mbb{R}^n\),
where \(Q \succeq 0\) and the constraint \(A \in \mbb{R}^{m \times n}\) is,
for our purposes, fat and full rank.

The \(\ell^2\) regularized version for \(\lambda > 0\) looks like
\begin{align}
  \text{minimize} &\quad \frac{1}{2} x^\top Q x + \lambda x^\top x
    \label{eqn:reg-qp} \\
  \text{subject to} &\quad Ax = b
\end{align}

\subsection{Karush-Kuhn-Tucker Conditions}
The Lagrangian for \eqref{eqn:qp} is
\begin{align*}
  L(x, p) = \frac{1}{2} x^\top Q x + p^\top (Ax - b)
\end{align*}
for which the
optimality conditions~\cite{boyd2004convex} are
\begin{align*}
  Q x^\star + A^\top p^\star = b,
    \qquad Ax^\star = b
\end{align*}
or more compcatctly expressed:
\begin{align}
  \begin{bmatrix} Q & A^\top \\ A & 0 \end{bmatrix}
  \begin{bmatrix} x^\star \\ p^\star \end{bmatrix}
  = \begin{bmatrix} 0 \\ b \end{bmatrix}
    \label{eqn:kkt-qp}
\end{align}

On the other hand the Lagrangian of the
regularized problem \eqref{eqn:reg-qp} is
\begin{align*}
  L_\lambda (x, p) = \frac{1}{2} x^\top (Q + \lambda I) x + p^\top (Ax - b)
\end{align*}
which has the KKT conditions
\begin{align}
  \begin{bmatrix} Q + \lambda I & A^\top \\ A & 0 \end{bmatrix}
  \begin{bmatrix} x_\lambda ^\star \\ p_\lambda ^\star \end{bmatrix}
    = \begin{bmatrix} 0 \\ b \end{bmatrix}
    \label{eqn:kkt-reg-qp}
\end{align}

\subsection{Matrix Inversion}
Consider a symmetric matrix
\begin{align*}
  Q = \begin{bmatrix} A & B \\ B^\top & C \end{bmatrix}
\end{align*}
when \(A \succ 0\), define
the Schur complement \(S = C - B^\top A^{-1} B\).
Then
\begin{align*}
  \begin{bmatrix} A & B \\ B^\top & C \end{bmatrix}^{-1}
    \begin{bmatrix}
      A^{-1} + A^{-1} B S^{-1} B^\top A^{-1} & - A^{-1} B S^{-1} \\
      - S^{-1} B^\top A^{-1} & S^{-1}
    \end{bmatrix}
\end{align*}


\section{Bounding Norms}
Noting that by linearity
\begin{align*}
  \begin{bmatrix} Q & A^\top \\ A & 0 \end{bmatrix}
    \begin{bmatrix} x^\star \\ p^\star \end{bmatrix}
  + \begin{bmatrix} \lambda I & 0 \\ 0 & 0 \end{bmatrix}
    \begin{bmatrix} x^\star \\ p^\star \end{bmatrix}
  = \begin{bmatrix} Q + \lambda I & A^\top \\ A & 0 \end{bmatrix}
    \begin{bmatrix} x^\star \\ p^\star \end{bmatrix}
  = \begin{bmatrix} \lambda x^\star \\ b \end{bmatrix}
\end{align*}
and so through subtracting equations,
\begin{align*}
  \begin{bmatrix} Q + \lambda I & A^\top \\ A & 0 \end{bmatrix}
    \begin{bmatrix}
      x^\star - x_{\lambda} ^\star \\ p^\star - p_\lambda ^\star
    \end{bmatrix}
    &= \begin{bmatrix} \lambda x^\star \\ 0 \end{bmatrix}
\end{align*}
Given our assumptions on \(Q + \lambda I \succ 0\) and \(A\) is full rank.
For ease of notation, define \(\Lambda = Q + \lambda I\).
The Schur complement is then
\(S = - A \Lambda^{-1} A^\top\), and
\begin{align*}
  \begin{bmatrix}
    x^\star - x_\lambda ^\star \\ p^\star - p_\lambda ^\star \end{bmatrix}
  = \begin{bmatrix}
      \Lambda^{-1} + \Lambda^{-1} A^\top S^{-1} A \Lambda^{-1}
        & - \Lambda^{-1} A^\top S^{-1} \\
      - S^{-1} A \Lambda^{-1} & S^{-1}
    \end{bmatrix}
    \begin{bmatrix}
      \lambda x^\star \\ 0
    \end{bmatrix}
\end{align*}
In other words:
\begin{align*}
  x^\star - x_\lambda ^\star
    = \parens{\Lambda^{-1} -
      \Lambda^{-1} A^\top (A \Lambda^{-1} A^\top)^{-1} A \Lambda^{-1}}
      \lambda x^\star
\end{align*}
To simplify notation slightly, use \(\Gamma = \Lambda^{-1}\)
because it looks like an upside-down \(L\).
Then
\begin{align*}
  x^\star - x_\lambda ^\star
    = \parens{\Gamma - \Gamma A^\top (A \Gamma A^\top)^{-1} A \Gamma)}
      \lambda x^\star
\end{align*}

\begin{theorem}
  Without loss of generality, assume that
  \begin{align*}
    Q = \diag(q_1, q_2, \ldots, q_n),
      \quad q_1 \geq \cdots \geq q_n \geq 0.
  \end{align*}
  Then for \(\lambda \leq q_1\),
  \begin{align*}
    \norm{x^\star - x_\lambda ^\star}
      \leq \frac{q_1 - q_n}{(q_1 + \lambda)^2} \lambda \norm{x^\star}
  \end{align*}
\end{theorem}
\begin{proof}
  Our angle of attack is to seek a bound
  \(\norm{\Gamma - \Gamma A^\top (A \Gamma A^\top)^{-1} A \Gamma}\).
  For this, note that both
  \(\Gamma \succ 0\)
  and \(\Gamma A^\top (A \Gamma A^\top)^{-1} A \Gamma \succ 0\)
  and so one technique is to find a matrix \(G\) such that
  \begin{align*}
    G \prec \Gamma A^\top (A \Gamma A^\top)^{-1} A \Gamma
      \quad \implies \quad
    \norm{\Gamma - \Gamma A^\top (A \Gamma A^\top)^{-1} A \Gamma}
      \leq \norm{\Gamma - G}
  \end{align*}
  We can iteratively construct \(G\).
  Let a singular value decomposition of \(A\) be
  \begin{align*}
    A = U \Sigma V^\top
      = U \begin{bmatrix} \Sigma_1 & 0 \end{bmatrix}
          \begin{bmatrix} V_1 ^\top \\ V_2 ^\top \end{bmatrix}
      = U \Sigma_1 V_1 ^\top,
      \qquad \Sigma_1 = \diag (\sigma_1, \sigma_2, \ldots, \sigma_m)
  \end{align*}
  Then first upper-bounding the inside of the inverse
  in order to lower-bound the overall inverse,
  \begin{align*}
    A \Gamma A^\top
      = U \Sigma_1 V_1 ^\top
          \begin{bmatrix}
            \frac{1}{q_1 + \lambda} & & \\
            & \ddots & \\
            & & \frac{1}{q_n + \lambda}
          \end{bmatrix}
          V_1 \Sigma_1 U^\top
      \preceq
        \frac{1}{q_n + \lambda} U \Sigma_1 ^2 U^\top
  \end{align*}
  Then for the inverse we would have
  \begin{align*}
    \Gamma A^\top (A \Gamma A^\top)^{-1} A \Gamma
      &\preceq
        \Gamma A^\top \bracks{(q_n +\lambda) U\Sigma_1 ^{-2} U^\top}A\Gamma \\
      &= \parens{q_n + \lambda}
          \Gamma V_1 \Sigma_1 U^\top
            \bracks{U \Sigma_1 ^{-2} U^\top}
            U \Sigma_1 V_1 ^\top \Gamma \\
      &\preceq \parens{q_n + \lambda} \Gamma \Gamma
  \end{align*}
  Putting these together,
  when \(\lambda \leq q_1\)
  \begin{align*}
    \Gamma - \Gamma A^\top (A \Gamma A^\top)^{-1} A \Gamma
      &\preceq \Gamma - (q_n + \lambda) \Gamma \Gamma \\
      &\preceq
        \begin{bmatrix}
        \frac{1}{q_1 + \lambda} & & \\ & \ddots & \\ & & \frac{1}{q_n + \lambda}
        \end{bmatrix}
        - (q_n + \lambda)
          \begin{bmatrix}
            \frac{1}{(q_1 + \lambda)^2} & & \\ & \ddots & \\ & & \frac{1}{(q_n + \lambda)^2}
          \end{bmatrix} \\
      &= \begin{bmatrix}
            \frac{q_1 + \lambda}{(q_1 + \lambda)^2}
              - \frac{q_n + \lambda}{(q_1 + \lambda)^2} & & \\
            & \ddots & \\
            & & \frac{q_n + \lambda}{(q_n + \lambda)^2}
                - \frac{q_n + \lambda}{(q_n + \lambda)^2}
          \end{bmatrix}
        \preceq \frac{q_1 -  q_n}{(q_1 + \lambda)^2} I
  \end{align*}
  Consequently, for the original bound
  \begin{align*}
    \norm{x^\star - x_\lambda ^\star}
      &= \norm{\parens{\Gamma - \Gamma A^\top (A \Gamma A^\top)^{-1} A \Gamma}
      \lambda x^\star} \\
    &\leq
      \norm{\Gamma - \Gamma A^\top (A \Gamma A^\top)^{-1} A \Gamma}
        \cdot \norm{\lambda x^\star} \\
    &\leq
      \frac{q_1 - q_n}{(q_1 + \lambda)^2}
        \lambda \norm{x^\star}
  \end{align*}

\end{proof}


\printbibliography

\end{document}

