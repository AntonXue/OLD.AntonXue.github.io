\documentclass[12pt]{article}

% Packages
\usepackage[margin=5em]{geometry} % 1 cm = 2.84528 em
\usepackage[backend=bibtex]{biblatex}
\bibliography{sources}
% \nocite{*}

\usepackage{lipsum}

% Paragraphs
\setlength{\parindent}{0em}
\setlength{\parskip}{1em}

% Includes
% Includes
\usepackage{amsfonts}
\usepackage{amsmath}
\usepackage{amssymb}
\usepackage{amsthm}
\usepackage{comment}
\usepackage[colorlinks]{hyperref}
\usepackage{letltxmacro}
\usepackage{listings}
\usepackage{stmaryrd}

% Spacing
\let\uspace\undefined
\newcommand{\uspace}{\ensuremath{\ }}

% Numbers
\let\N\undefined
\newcommand{\N}{\ensuremath{\mathbb{N}}}

\let\Z\undefined
\newcommand{\Z}{\ensuremath{\mathbb{Z}}}

\let\Zz\undefined
\newcommand{\Zz}{\ensuremath{\mathbb{Z}^{\geq 0}}}

\let\Zp\undefined
\newcommand{\Zp}{\ensuremath{\mathbb{Z}^{+}}}

\let\Q\undefined
\newcommand{\Q}{\ensuremath{\mathbb{Q}}}

\let\R\undefined
\newcommand{\R}{\ensuremath{\mathbb{R}}}

\let\Rz\undefined
\newcommand{\Rz}{\ensuremath{\mathbb{R}^{\geq 0}}}

\let\Rp\undefined
\newcommand{\Rp}{\ensuremath{\mathbb{R}^{+}}}

\let\C\undefined
\newcommand{\C}{\ensuremath{\mathbb{C}}}

\let\mcal\undefined
\newcommand{\mcal}[1]{\ensuremath{\mathcal{#1}}}

\let\mbb\undefined
\newcommand{\mbb}[1]{\ensuremath{\mathbb{#1}}}

% Grouping
\let\parens\undefined
\newcommand{\parens}[1]{\ensuremath{\left(#1\right)}}

\let\brackets\undefined
\newcommand{\brackets}[1]{\ensuremath{\left[#1\right]}}

\let\braces\undefined
\newcommand{\braces}[1]{\ensuremath{\left\{#1\right\}}}

\let\angles\undefined
\newcommand{\angles}[1]{\ensuremath{\left\langle#1\right\rangle}}

\let\ceil\undefined
\newcommand{\ceil}[1]{\ensuremath{\left\lceil#1\right\rceil}}

\let\floor\undefined
\newcommand{\floor}[1]{\ensuremath{\left\lfloor#1\right\rfloor}}

% Sets and Function Spaces
\let\type\undefined
\newcommand{\type}[3]{\ensuremath{#1 \colon #2 \to #3}}

\let\powset\undefined
\newcommand{\powset}[1]{\ensuremath{2^{#1}}}

\let\mod\undefined
\newcommand{\mod}{\ensuremath{\uspace\mathrm{mod}\uspace}}

\let\ker\undefined
\newcommand{\ker}{\ensuremath{\mathrm{ker}}}

\let\dom\undefined
\newcommand{\dom}{\ensuremath{\mathrm{dom}}}

\let\ran\undefined
\newcommand{\ran}{\ensuremath{\mathrm{ran}}}

\let\im\undefined
\newcommand{\im}{\ensuremath{\mathrm{im}}}

\let\coker\undefined
\newcommand{\coker}{\ensuremath{\mathrm{coker}}}

\let\codim\undefined
\newcommand{\codim}{\ensuremath{\mathrm{codim}}}

% Complex Numbers
\let\Re\undefined
\newcommand{\Re}{\ensuremath{\mathrm{Re}}}

\let\Im\undefined
\newcommand{\Im}{\ensuremath{\mathrm{Im}}}

% Probability
\let\Pr\undefined
\newcommand{\Pr}{\ensuremath{\mathrm{Pr}}}

\let\E\undefined
\newcommand{\E}{\ensuremath{\mathrm{E}}}

\let\Var\undefined
\newcommand{\Var}{\ensuremath{\mathrm{Var}}}

\let\Cov\undefined
\newcommand{\Cov}{\ensuremath{\mathrm{Cov}}}

% Integrals
\let\dee\undefined
\newcommand{\dee}[1]{\ensuremath{\uspace d #1}}

% Linear Algebra
\let\abs\undefined
\newcommand{\abs}[1]{\ensuremath{\left\lvert#1\right\rvert}}

\let\norm\undefined
\newcommand{\norm}[1]{\ensuremath{\left\lVert#1\right\rVert}}


% Computational Complexity
\let\class\undefined % Complexity class
\newcommand{\class}[1]{\ensuremath{\mathbf{#1}}}

\let\prob\undefined % Complexity problem
\newcommand{\prob}[1]{\ensuremath{\text{#1}}}

% Theorem environment
\theoremstyle{plain}
\newtheorem{theorem}{Theorem}

\theoremstyle{plain}
\newtheorem{lemma}{Lemma}

\theoremstyle{plain}
\newtheorem{claim}{Claim}

\theoremstyle{plain}
\newtheorem{fact}{Fact}

\theoremstyle{plain}
\newtheorem{remark}{Remark}

\theoremstyle{plain}
\newtheorem{definition}{Definition}

\theoremstyle{plain}
\newtheorem{example}{Example}

\theoremstyle{plain}
\newtheorem{question}{Question}

% Code environment
\lstdefinestyle{plainsty}{
  basicstyle=\small\ttfamily,
  language=C,
  xleftmargin=\parindent,
  aboveskip=1em,
  belowskip=1em,
  showspaces=false,
  showstringspaces=false,
  keywordstyle = {},
}

\lstnewenvironment{pcode}{\lstset{style=plainsty}}{}

\let\pinl\undefined
\newcommand{\pinl}{\lstinline[style=plainsty]}

\newcommand*{\SavedLstInline}{} % Allows plain code usage in math mode.
\LetLtxMacro\SavedLstInline\pinl
\DeclareRobustCommand*{\pinl}{%
  \ifmmode
    \let\SavedBGroup\bgroup
    \def\bgroup{%
      \let\bgroup\SavedBGroup
      \hbox\bgroup
    }%
  \fi
\SavedLstInline}

\let\ttcode\undefined
\newcommand{\ttcode}[1]{\small{\texttt{#1}}}


% Text markings

\let\tturl\undefined
\newcommand{\tturl}[1]{\href{#1}{\texttt{#1}}}

\let\red\undefined
\newcommand{\red}[1]{\textbf{\color{red}#1}}


% Abbreviations
\let\st\undefined
\newcommand{\st}{\ensuremath{\uspace\colon\uspace}}

\let\ow\undefined
\newcommand{\ow}{\ensuremath{\text{otherwise}}}


% \usepackage{stmaryrd}


\let\usp\undefined
\newcommand{\usp}{\>}

% Grouping
\let\parens\undefined
\newcommand{\parens}[1]{\text{(}#1\text{)}}

\let\brackets\undefined
\newcommand{\brackets}[1]{\text{[}#1\text{]}}

\let\braces\undefined
\newcommand{\braces}[1]{\text{\{}#1\text{\}}}


% Language

\let\regex\undefined
\newcommand{\regex}{\ensuremath{E}}

\let\regexes\undefined
\newcommand{\regexes}{\ensuremath{\mathcal{E}}}

\let\reglang\undefined
\newcommand{\reglang}{\ensuremath{L_{R}}}

\let\reglangs\undefined
\newcommand{\reglangs}{\ensuremath{\mathcal{L}_R}}

\let\lang\undefined
\newcommand{\lang}{\ensuremath{L}}

\let\langs\undefined
\newcommand{\langs}{\ensuremath{\mathcal{L}}}

\let\alphabet\undefined
\newcommand{\alphabet}{\ensuremath{\Sigma}}

\let\empstr\undefined
\newcommand{\empstr}{\ensuremath{\varepsilon}}

\let\langmin\undefined
\newcommand{\langmin}{\ensuremath{\eta}}


% Measures

\let\mucount\undefined
\newcommand{\mucount}{\ensuremath{\mu_{c}}}

\let\premucount\undefined
\newcommand{\premucount}{\ensuremath{\mu_{0, c}}}






% Author
\title{Separating Strings with Automata and SAT}
% \author{Anton Xue}
% \date{\today}
\date{}

% Document
\begin{document}
\maketitle

\section{Introduction}
In this sketch we study
how to use SAT to construct a non-deterministic finite automata (NFA)
that separates two finite sets of strings.
This yields a decision procedure for calculating
the metric distance between two regular sets with respect to the
(inverse) automata size metric.


\section{Preliminaries}
An alphabet \(\Sigma\) is a finite set of unique symbols.
Let \(\Sigma^\star\) denote the set of all finite strings consisting of
letters from \(\Sigma\).
A language \(\mcal{L} \subseteq \Sigma^\star\) is then
a countable set of strings of \(\Sigma\).
Often \(\Sigma\) is implicitly assumed.

A non-deterministic finite automata (NFA) \(\mcal{A}\) is represented
as a tuple
\(\mcal{A} = \parens{\Sigma, Q, \Delta, S, F}\)
where \(\Sigma\) is a finite alphabet,
\(Q\) is a finite set of states,
\(\type{\Delta}{\parens{Q \times \Sigma}}{\powset{Q}}\)
is the transition function,
\(S \subseteq Q\) is the inital states,
and \(F \subseteq Q\) is a set of final states.

For some string \(w\) we say that \(\mcal{A}\) accepts \(w\)
if there exists a sequence of transitions on which \(\mcal{A}\) ends
in a final state when reading \(w\).
We abuse notation and say
that \(\mcal{A}\) accepts \(w\) if \(\mcal{A}\parens{w} = 1\).
Similarly, we say \(\mcal{A}\parens{w} = 0\) if \(\mcal{A}\) rejects \(w\).

Let \(w_i\) denote the \(i\)th letter of the string \(w\),
and \(\varepsilon\) be the empty string.



\section{Separating Sets}

The question that this sketch attempts to answer can be formalized as follows:

\begin{question}
Given a positive set \(P \subseteq \Sigma^\star\)
and negative set \(N \subseteq \Sigma^\star\)
with \(P\) and \(N\) disjoint,
does there eixst an NFA
\(\mcal{A} = \parens{\Sigma, Q, \Delta, S, F}\)
with \(\abs{Q} = n\)
such that for all \(u \in P\) in the positive set \(\mcal{A}\parens{u} = 1\),
but for all \(v \in N\) in the negative set \(\mcal{A}\parens{v} = 0\).
\end{question}

We achieve this by constructing a boolean satisfiability formula
that encodes \(P\) and \(N\),
and is satisfiable if and only if such \(\mcal{A}\) exists.
There are several high-level insights that we leverage:
\begin{enumerate}
  \item[(1)]
    NFAs can be represented as directed multi-edge graphs
    where each edge is labeled by one letter from \(\Sigma\).
    In other words, let \(e_{i, j, \sigma}\) be an indicator
    variable encodes the indicator of a transition from state \(q_i\)
    to state \(q_j\) on the letter \(\sigma\).

  \item[(2)]
    For each \(u \in P\),
    we can create a formula that forces a sequence of edge walks
    resulting in a final state in \(\mcal{A}\).
    Similarly for each \(v \in N\) we can encode a sequence that will force
    a rejection of \(v\) in \(\mcal{A}\).

\end{enumerate}

For any \(w \in \Sigma^\star\), consider the following encoding:
\begin{align*}
  \pi_w \equiv
    \bigwedge_{1 \leq t < \abs{w}}
      \parens{\bigvee_{i \neq j \neq k}
        e_{i, j, w_t} \land e_{j, k, w_{t + 1}}}
\end{align*}
This forces the existence of a path on \(\mcal{A}\) for \(w\).
Additionally, to force \(w\) to stop on a final state,
we may either have every state be a final state,
or just set the following formula:
\begin{align*}
  \varphi_w \equiv
    \brackets{\bigwedge_{i \neq j} \parens{e_{i, j, w_{1}} \implies s_i}}
      \land
    \brackets{\bigwedge_{i \neq j} \parens{e_{i, j, w_{\abs{w}}} \implies f_j}}
\end{align*}
Essentially, this forces each \(w\) to begin processing on an initial state
as indicated by \(s_i\), and end on a final state as indicated by \(f_j\).
Thus, the conjunction \(\pi_w \land \varphi_w\) will be true if and
only if \(\mcal{A}\) accepts \(w\).
Hence, the satisfaction of
its negation will then force \(\mcal{A}\) to reject \(w\).
We leverage this in order to define
over \(P\) and \(N\):
\begin{align*}
  \Phi_{P, N} =
    \brackets{\bigwedge_{u \in P} \parens{\pi_u \land \varphi_u}}
      \land
    \brackets{\bigwedge_{v \in N} \neg\parens{\pi_v \land \varphi_v}}
\end{align*}
Thus, \(\Phi_{P, N}\) defines a formula that will make
\(\mcal{A}\) to accept \(P\) and reject \(N\).

The variables of the \(\Phi_{P, N}\)
are the indicators for edges of form \(e_{i, j, \sigma}\) and
final states of form \(f_j\).
Suppose that \(\Sigma\) is known.
If \(\Phi_{P, N}\) is satisfiable,
then \(\mcal{A} = \parens{\Sigma, Q, \Delta, S, F}\) can be extracted as:
\begin{align*}
  Q &= \braces{q_1, \ldots, q_n} \\
  \Delta &= \braces{\parens{\parens{q_i, \sigma}, q_j} \st e_{i, j, \sigma} = \top} \\
  S &= \braces{q_i \st s_i = \top} \\
  F &= \braces{q_j \st f_j = \top} \\
\end{align*}
Note that \(\Delta\) is slightly overloaded.



\end{document}


