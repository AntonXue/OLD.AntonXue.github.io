\documentclass[12pt]{article}

% Packages
\usepackage[margin=5em]{geometry} % 1 cm = 2.84528 em
\usepackage[backend=bibtex]{biblatex}
\bibliography{sources}
% \nocite{*}

\usepackage{lipsum}

% Paragraphs
\setlength{\parindent}{0em}
\setlength{\parskip}{1em}

% Includes
\input{antonxue-lib.tex}
\input{str-lib.tex}


% Author
\title{String Set Metrics}
% \author{Anton Xue}
% \date{\today}
\date{}

% Document
\begin{document}
\maketitle

%%%%%%%%%%%%%%%%%%%%%%%%%%%%%%%%%%%%%%%%%%%%%%%%%%%%%%%%%%%%
\section{Introduction}
In this sketch we are interested in studying metric spaces between
sets of strings.

%%%%%%%%%%%%%%%%%%%%%%%%%%%%%%%%%%%%%%%%%%%%%%%%%%%%%%%%%%%%
\section{Preliminaries}

\begin{definition}[Metric Space]
  A metric space \(\parens{M, d}\) is a set \(M\) along with a distance
  function \(\type{d}{M \times M}{\Rz}\) such that
  for any \(x, y, z \in M\):
  \begin{enumerate}
    \item[(1)]
      \(d \parens{x, y} \geq 0\)

    \item[(2)]
      \(d\parens{x, y} = 0 \iff x = y\)

    \item[(3)]
      \(d\parens{x, y} = d\parens{y, x}\)

    \item[(4)]
      \(d\parens{x, z} \leq d\parens{x, y} + d\parens{y, z}\)
  \end{enumerate}
\end{definition}

\begin{definition}[Alphabet]
  An alphabet \(\Sigma\) is a finite set of unique symbols.
\end{definition}

\begin{definition}[String]
  Given an alphabet \(\Sigma\),
  a string \(\sigma\) is a finite sequence of symbols from \(\Sigma\).
\end{definition}

\begin{definition}[Alphabet Strings]
  Let \(\Sigma^\star\) denote the set of all possible strings from \(\Sigma\).
\end{definition}

\begin{definition}[String Metric]
  A function \(\type{\delta}{\Sigma^\star \times \Sigma^\star}{\Rz}\)
  that satisfies metric space axioms.
\end{definition}


\section{String Set Metric Spaces}


\subsection{Hausdorff Metric}
We first consider the following problem.
Given a single string \(\sigma\), and a set of string \(A\),
how might we calculate a distance from \(\sigma\) to \(A\)?
Let \(\delta\) be a string metric, then one idea is as follows:

\begin{definition}[Merge Distance]
  The merge distance \(\type{\delta_{M}}{\Sigma \times \Sigma^\star}{\Rz}\)
  is defined as:
  \begin{align*}
    \delta_M \parens{\sigma, A}
      = \inf_{a \in A} \braces{\delta\parens{\sigma, a} \st a \in A}
  \end{align*}
\end{definition}

The idea here is that we take the string in \(A\) that most closely
resembles \(\sigma\) with respect to the string metric \(\delta\),
and consider that the distance between \(\sigma\) and \(A\).

We can take this idea further.
Suppose we have two sets of strings \(A\) and \(B\).
The Hausdorff distance is then defined as follows:

\begin{definition}[Hausdorff Distance]
  The Hausdorff distance
  \(\type{d_H}{\Sigma^\star \times \Sigma^\star}{\Rz}\) is defined as:
  \begin{align*}
    d_H \parens{A, B} =
      \max \braces{\sup \braces{\delta_{M} \parens{a, B} \st a \in A},
                   \sup \braces{\delta_{M} \parens{b, A} \st b \in B}}
  \end{align*}
\end{definition}
The Hausdorff distance is essentially the promotion of a
metric space \(\parens{X, d}\) to another metric space
\(\parens{\powset{X}, d_H}\) where \(d_H\) is defind with respect to \(d\).
The definition is not unique to strings here.


\pagebreak

\printbibliography

\end{document}

