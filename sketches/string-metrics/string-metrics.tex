\documentclass[12pt]{article}

% Packages
\usepackage[margin=5em]{geometry} % 1 cm = 2.84528 em
\usepackage[backend=bibtex]{biblatex}
\bibliography{sources}
% \nocite{*}

\usepackage{lipsum}

% Paragraphs
\setlength{\parindent}{0em}
\setlength{\parskip}{1em}

% Includes
\input{antonxue-lib.tex}
\input{str-lib.tex}


% Author
\title{String Set Metrics}
% \author{Anton Xue}
% \date{\today}
\date{}

% Document
\begin{document}
\maketitle

%%%%%%%%%%%%%%%%%%%%%%%%%%%%%%%%%%%%%%%%%%%%%%%%%%%%%%%%%%%%
\section{Introduction}
In this sketch we are interested in studying metric spaces between
sets of strings.

%%%%%%%%%%%%%%%%%%%%%%%%%%%%%%%%%%%%%%%%%%%%%%%%%%%%%%%%%%%%
\section{Preliminaries}

\begin{definition}[Metric Space]
  A metric space \(\parens{M, d}\) is a set \(M\) along with a distance
  function \(\type{d}{M \times M}{\Rz}\) such that
  for any \(x, y, z \in M\):
  \begin{enumerate}
    \item[(1)]
      \(d \parens{x, y} \geq 0\)

    \item[(2)]
      \(d\parens{x, y} = 0 \iff x = y\)

    \item[(3)]
      \(d\parens{x, y} = d\parens{y, x}\)

    \item[(4)]
      \(d\parens{x, z} \leq d\parens{x, y} + d\parens{y, z}\)
  \end{enumerate}
\end{definition}

\begin{definition}[Alphabet]
  An alphabet \(\Sigma\) is a finite set of unique symbols.
\end{definition}

\begin{definition}[String]
  Given an alphabet \(\Sigma\),
  a string \(\sigma\) is a finite sequence of symbols from \(\Sigma\).
\end{definition}

\begin{definition}[Alphabet Strings]
  Let \(\Sigma^\star\) denote the set of all possible strings from \(\Sigma\).
\end{definition}

\begin{definition}[String Metric]
  A function \(\type{\delta}{\Sigma^\star \times \Sigma^\star}{\Rz}\)
  that satisfies metric space axioms.
\end{definition}


\section{String Set Metric Spaces}

\subsection{Merging Sets}
We first consider the following problem.
Given a single string \(\sigma\), and a set of string \(A\),
how might we calculate a distance from \(\sigma\) to \(A\)?
Let \(\delta\) be a string metric, then one idea is as follows:
\begin{align*}
  d\parens{\sigma, A} = \inf \braces{\delta \parens{\sigma, a} \st a \in A}
\end{align*}
The idea here is that we take the string in \(A\) that most closely
resembles \(\sigma\) with respect to the string metric \(\delta\),
and consider that the distance between \(\sigma\) and \(A\).

We can take this idea further.
Suppose we have two sets of strings \(A\) and \(B\).
Let us define the merge cost of \(A\) into \(B\) as follows
with the
\(\type{\gg}{\powset{\Sigma^\star} \times \powset{\Sigma^\star}}{\Rz}\)
function:
\begin{align*}
  A \gg B = \sum_{a \in A} \inf \braces{\delta\parens{a, b} \st b \in B}
\end{align*}
Again, for each \(a \in A\), we find their individual merge cost into \(B\).
Of course one additional possibility is weighting strings by length
instead of just purely summing them.
But most importantly, it sure feels great to make up notation!

\subsection{Bi-Directional Merging}
Let's try the following:
\begin{definition}[Bidirectional Merge Cost]
  For set of strings \(A, B \subseteq \powset{\Sigma^\star}\),
  define the bi-directional merge cost
  \(\type{M}{\powset{\Sigma^\star} \times \powset{\Sigma^\star}}{\Rz}\)
  as follows:
  \begin{align*}
    M\parens{A, B} = \parens{A \gg B} + \parens{B \gg A}
  \end{align*}
\end{definition}
For now we are only concerned about when \(A\) and \(B\) are both finite sets.
Later we can try to use probability distribution style weighting
to account for when \(A\) and \(B\) are infinite.
Nevertheless:

\begin{theorem}
  The bi-directional merge cost \(M\) is a metric.
\end{theorem}
\begin{proof}
  We prove only the triangle inequality.
  Consider some \(A, B, C \subseteq \powset{\Sigma^\star}\) and:
  \begin{align*}
    M\parens{A, C}
      &= \underbrace{
          \sum_{a \in A} \inf \braces{\delta\parens{a, c} \st c \in c}
        }_{S_{A, C}} +
        \underbrace{
          \sum_{c \in C} \inf \braces{\delta\parens{c, a} \st a \in A}
        }_{S_{C, A}} \\
    M\parens{A, B}
      &=
        \underbrace{
          \sum_{a \in A} \inf \braces{\delta\parens{a, b} \st b \in B}
        }_{S_{A, B}} +
        \underbrace{
          \sum_{b \in B} \inf \braces{\delta\parens{b, a} \st a \in A}
        }_{S_{B, A}} \\
    M\parens{B, C}
      &=
        \underbrace{
          \sum_{b \in B} \inf \braces{\delta\parens{b, c} \st c \in C}
        }_{S_{B, C}} +
        \underbrace{
          \sum_{c \in C} \inf \braces{\delta\parens{c, b} \st b \in B}
        }_{S_{C, B}}
  \end{align*}
  Consider some pair \(\parens{a, c}\) used in the sum \(S_{A, C}\).
  Observe that there then exists some \(b_1\) and \(b_2\) such that
  \(\parens{a, b_1}\) appears in the sum \(S_{A, B}\) and
  \(\parens{c, b_2}\) appears in the sum \(S_{C, B}\).
  Since \(\delta\) is a string metric, this naturally means that:
  \begin{align*}
    \delta\parens{a, c} \leq
      \delta\parens{a, b_1} + \delta\parens{b_1, b_2} + \delta\parens{b_2, c}
  \end{align*}
  Furthermore, we can uniquely identify
  the pairs \(\parens{a, b_1}\) and \(\parens{c, b_2}\) with
  \(\parens{a, c}\),
  since both \(a\) and \(c\) are iterated on in \(S_{A, B}\) and \(S_{C, B}\)
  respectively.
  Therefore, every such pair in \(S_{A, C}\) can be uniquely identified
  with two such sets in \(S_{A, B}\) and \(S_{B, C}\).
  The consequence is then:
  \begin{align*}
    \underbrace{
      \sum_{a \in A} \inf\braces{\delta\parens{a, c} \st c \in C}
    }_{S_{A, C}}
      \leq
      \underbrace{
        \sum_{a \in A} \inf\braces{\delta\parens{a, b} \st b \in B}
      }_{S_{A, B}} +
      \underbrace{
        \sum_{c \in A} \inf\braces{\delta\parens{c, b} \st b \in B}
      }_{S_{C, B}}
  \end{align*}
  We mirror the argument for \(S_{C, A}\),
  to get another inequality:
  \begin{align*}
    \underbrace{
      \sum_{c \in C} \inf\braces{\delta\parens{c, a} \st a \in A}
    }_{S_{C, A}}
      \leq
      \underbrace{
        \sum_{c \in C} \inf\braces{\delta\parens{c, b} \st b \in B}
      }_{S_{C, B}} +
      \underbrace{
        \sum_{a \in A} \inf\braces{\delta\parens{a, b} \st a \in A}
      }_{S_{A, B}}
  \end{align*}
  Collectively this results in the triangle inequality:
  \begin{align*}
    M\parens{A, C} \leq M\parens{A, B} + M\parens{B, C}
  \end{align*}
  
\end{proof}



\pagebreak

\printbibliography

\end{document}

