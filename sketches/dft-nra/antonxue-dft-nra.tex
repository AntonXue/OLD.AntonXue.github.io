\documentclass[12pt]{article}

% Packages
\usepackage[margin=6em]{geometry} % 1 cm = 2.84528 em
\usepackage[colorlinks=true,linkcolor=red,urlcolor=blue]{hyperref}
\usepackage{lipsum}

% Paragraphs
\setlength{\parindent}{0em}
\setlength{\parskip}{1em}

% Includes
\input{antonxue-lib}

\let\H\undefined
\newcommand{\H}{\mathcal{H}}

\let\F\undefined
\newcommand{\F}{\mathcal{F}}

% Author
\author{}
\title{Discrete Fourier Transforms for Non-linear Real Arithmetics}
% \date{\today}
\date{}

% Document
\begin{document}
\maketitle

\subsection*{General Fourier Transform}
Consider the Hilbert space $\H = L^2 \parens{\brackets{0, 1}}$ endowed with
the orthonormal bases $\braces{e_n}_{n \in \Z}$, where we define
$e_n \parens{x} = e^{i 2\pi n x}$.
A Fourier transform $\type{\F_n}{\H}{\C}$ can be defined as the inner product:

\begin{align*}
  \F_n\brackets{f}
    = \inner{f, e_n}
    = \int_0^1 f\parens{x} \overline{e_n \parens{x}} dx
    = \int_0^1 f\parens{x} e^{-i 2 \pi n x} dx
\end{align*}

By linearity of the integral, each Fourier transform $\F_n$
can be seen as a linear functional.
We make a stronger claim, that the Fourier transform is continuous with
norm $1$:

\begin{align*}
  \norm{F_n}_{\adj{\H}}
    = \sup_{\norm{f}_{\H} = 1}
        \abs{\int_0^1 f\parens{x} \overline{e_n \parens{x}} dx}
    \leq \parens{\int_0^1 \abs{f\parens{x}}^2 dx}^{1/2}
              \parens{\int_0^1 \abs{e_n \parens{x}}^2 dx}^{1/2}
    = \norm{f}_{\H} \cdot \norm{e_n}_{H}
\end{align*}
The inequality is sharp when $f = e_{n}$.
We write the partial Fourier series $\type{S_N}{\H}{\H}$ as:

\begin{align*}
  S_N \brackets{f}
    = \sum_{n = -N}^{N} \inner{f, e_n}
    = \sum_{n = -N}^{N} \int_0^1 f(x) e^{- i 2 \pi n x} dx
\end{align*}

We omit the proof, but do note that this converges in norm for any
$f \in \mathcal{H}$:
\begin{align*}
  \lim_{N \to \infty} \norm{f - S_N \brackets{f}}_{\H}
    = \lim_{N \to \infty}
      \abs{f\parens{x} -
            \sum_{n = -N}^{N} \int_0^1 f\parens{x} e^{-i 2\pi nx} dx}
    \to 0
\end{align*}

\subsection*{Discrete Fourier Transform}



\end{document}

