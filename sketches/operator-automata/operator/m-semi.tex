
\subsection{\(M\)-Semimodules and Operator Norms}
Let \(\Sigma\) be a finite set and
\(M\) to be the free monoid finitely generated by \(\Sigma\):
\begin{align*}
  M = \parens{\Sigma, \cdot, \one}
\end{align*}

\begin{definition}[Normed Monoid]
  A normed monoid is a monoid \(M\) equipped with norm \(\norm{\cdot}_M\).
\end{definition}

Because monoids lack addition,
such norms only concern multiplication.

We may extend monoids in general to define the notion of a \(M\)-semiring
with respect to a monoid \(M\).
In particular, \(M\)-semirings are generated by the power set of \(M\),
\(\powset{M}\).

\begin{definition}[\(M\)-Semiring]
  Let \(M\) be a monoid.
  A \(M\)-semiring \(R_M\) is a semiring:
  \begin{align*}
    R_M = \parens{\powset{M}, \cup, \cdot, \zero, \one}
  \end{align*}
  Where semiring addition is set union \(\cup\) with identity \(\zero\),
  and semiring multiplication \(\cdot\) and identity \(\one\)
  are carried over from \(M\).
\end{definition}

We may write \(R_M\) as just \(R\) when context is clear.
Sometimes we will also just call \(M\)-semirings as semirings,
because they are just special cases of semirings.
As with the case of normed monoids,
we may extend this to normed \(M\)-semirings:

\begin{definition}[\(M\)-Semiring \(p\)-Norm]
  For \(1 \leq p \leq \infty\),
  a \(p\)-normed \(M\)-semiring is \(R_M\)
  equipped with a norm \(\norm{\cdot}_{R_M, p}\), defined as:
  \begin{align*}
    \forall x \in R_M \st
      \norm{x}_{R_M, p} = \parens{\sum_{a \in R_M} \norm{a}_M ^p}^{1/p}
  \end{align*}
\end{definition}

Observe that because \(M\) is finitely generated,
each \(x \in R_M\) is therefore countable,
and hence so is the sum.
When \(p = \infty\), this is just a \(\sup\) norm.
Similar definitions can be found in literature~\cite{kudlek2000lemmata}.

Extending \(M\)-semirings,
we define \(\parens{M, n}\)-semimodules:

\begin{definition}[\(\parens{M, n}\)-Semimodule]
  A \(\parens{M, n}\)-semimodule \(R_M ^n\)
  is a semimodule consisting of \(n\) isomorphic copies of \(R_M\).
\end{definition}

If \(x \in R_M ^n\), write \(x_i\) to denote the \(i\)th element from
\(R_M\) in some canonical representation of \(R_M ^n\).
Often this is just a row (horizontal) or column (vertical)
vector of length \(n\).

We write \(R^n\) instead of \(R_M ^n\) when \(M\) is understood from context.
Again, we extend norms to \(\parens{M, n}\)-semimodules:

\begin{definition}[\(\parens{M, n}\)-Semimodule \(\parens{p, q}\)-Norm]
  Let \(R_M ^n\) be a \(\parens{M, n}\)-semimodule
  and take \(1 \leq p, q \leq \infty\).
  Define the \(\parens{M, n}\)-semimodule \(\parens{p, q}\)-norm
  \(\norm{\cdot}_{R_M ^n, p, q}\):
  \begin{align*}
    \forall x \in R_M ^n \st
      \norm{x}_{R_M ^n, p, q}
        = \parens{\sum_{i = 1}^{n} \norm{x_i}_{R_M, p}^q}^{1 / q}
  \end{align*}
\end{definition}

When \(p = q = \infty\), these are just the \(\sup\)-norm.
Often we may only care about the case of \(p = \infty\) and \(q = 1\),
which is the \(\sup\)-norm over each \(R_M\),
and the \(1\)-norm over the \(n\) copies of \(R_M\) in \(R_M ^n\).

With the notion of norms, we again extend such notions to
linear operators between some \(R_M ^n\) and \(R_M ^m\).

\begin{definition}[Linear Operator Norm]
  Take \(R_M ^n\) equipped with norm \(\norm{\cdot}_{R_M ^n, p_n, q_n}\)
  and \(R_M ^m\) equipped with norm \(\norm{\cdot}_{R_M ^m, p_m, q_m}\).
  Let \(\mcal{L}\) be the space of linear operators between
  \(R_M ^n\) and \(R_M ^m\),
  then define the \(\norm{\cdot}_{\mcal{L}}\) as:
  \begin{align*}
    \forall T \in \mcal{L} \st
      \norm{T}_{\mcal{L}} =
        \inf \braces{c \in \Rz \st
                \norm{Tx}_{R_M ^m, p_m, q_m}
                  \leq c \norm{x}_{R_M ^n, p_n, q_n},
                \forall x \in R_M ^n}
  \end{align*}
\end{definition}


