
\subsection{\(M\)-Semimodules and Operator Norms}
Let \(\Sigma\) be a finite set and
\(M\) to be the monoid finitely generated by \(\Sigma\):
\begin{align*}
  M = \parens{\Sigma, \cdot, \one}
\end{align*}

\begin{definition}[Normed Monoid]
  A normed monoid is a monoid \(M\) equipped with a norm
  \(\type{\norm{\cdot}}{M}{\Rz}\).
\end{definition}

Because monoids lack addition,
such norms only concern multiplication.

We may extend monoids in general to define the notion of a \(M\)-semiring.
In particular,
\(M\)-semirings are defined with respect to a particular monoid \(M\),
and are generated from its power set \(\powset{M}\).

\begin{definition}[\(M\)-Semiring]
  Let \(M\) be a finitely generated monoid.
  A \(M\)-semiring \(R\) is a semiring such that:
  \begin{align*}
    R = \parens{\powset{M}, \cup, \cdot, \zero, \one}
  \end{align*}
  Where semiring addition is set union \(\cup\) with identity \(\zero\),
  and semiring multiplication \(\cdot\) and identity \(\one\)
  are carried over from \(M\).
\end{definition}

As with the case of normed monoids,
we may extend this to normed \(M\)-semirings.
In particular, we pay special attention to \(p\)-norms.

\begin{definition}[\(M\)-Semiring \(p\)-Norm]
  Let \(M\) be a finitely generated and normed monoid.
  For \(1 \leq p \leq \infty\),
  a \(p\)-normed \(M\)-semiring is \(R\)
  is equipped with a norm \(\type{\norm{\cdot}_p}{R}{\Rz}\):
  \begin{align*}
    \norm{x}_{p} = \parens{\sum_{a \in x} \norm{a} ^p}^{1/p}
  \end{align*}
\end{definition}

The sum here utilizes the monoid norm.
Observe that because \(M\) is finitely generated,
each \(x \in R_M\) is therefore countable,
and hence so is the sum.
When \(p = \infty\), this is just a \(\sup\) norm.
Similar definitions can be found in literature~\cite{kudlek2000lemmata}.

Extending \(M\)-semirings,
we define \(\parens{M, n}\)-semimodules:

\begin{definition}[\(\parens{M, n}\)-Semimodule]
  A \(\parens{M, n}\)-semimodule \(R ^n\)
  is a free semimodule generated by \(n\) isomorphic copies of
  the \(M\)-semiring \(R\).
\end{definition}

If \(x \in R ^n\), write \(x_i\) to denote the \(i\)th element from
\(R\) in some canonical representation of \(R ^n\).
Often this is just a row (horizontal) or column (vertical)
vector of length \(n\).

Again, we extend norms to \(\parens{M, n}\)-semimodules:

\begin{definition}[\(\parens{M, n}\)-Semimodule \(\parens{p, q}\)-Norm]
  Let \(R\) be a \(p\)-normed \(M\)-semiring.
  Let \(R ^n\) be a \(\parens{M, n}\)-semimodule
  and take \(1 \leq p, q \leq \infty\).
  A \(\parens{p, q}\)-normed \(\parens{M, n}\)-semimodule
  is a semimodule with norm \(\type{\norm{\cdot}_{p, q}}{R^n}{\Rz}\):
  \begin{align*}
    \norm{x}_{p, q}
      = \parens{\sum_{i = 1}^{n} \norm{x_i}_{p}^q}^{1 / q}
  \end{align*}
\end{definition}

When \(p = q = \infty\), these are just the \(\sup\)-norm.
Often we may only care about the case of \(p = \infty\) and \(q = 1\),
which is the \(\sup\)-norm over each \(R\),
and the \(1\)-norm over the \(n\) copies of \(R\) in \(R^n\).

With the notion of norms, we again extend such notions to
linear operators mapping from \(R^n\) to \(R^m\).

\begin{definition}[Linear Operator Norm]
  Let \(\mcal{L}\) be the space of linear operators mapping
  a \(\parens{M, n}\)-semimodule \(R^n\) with norm \(\norm{\cdot}_{R^n}\)
  to a \(\parens{M, m}\)-semimodule \(R^m\) with norm \(\norm{\cdot}_{R^m}\).
  Then define the operator norm
  \(\type{\norm{\cdot}_{\mcal{L}}}{\mcal{L}}{\Rz}\) as:
  \begin{align*}
    \norm{T}_{\mcal{L}} =
      \inf \braces{c \in \Rz \st
              \norm{Tx}_{R^m}
                \leq c \norm{x}_{R^n},
              \forall x \in R^n}
  \end{align*}
\end{definition}


