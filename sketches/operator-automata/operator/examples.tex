
\subsection{Examples}

\begin{example}
  Consider \(\Sigma = \braces{a, b, c}\),
  and a finite automata below:

  \begin{center}
  \begin{tikzpicture}
    [->,
     >=stealth',
     shorten >=1pt,
     auto,
     node distance=2cm,
     semithick,
     state/.style={circle, draw, minimum size=1cm} 
    ]
    \node[state] (Q1) at (0, 0) {\(q_1\)};
    \node[state] (Q2) at (3, 1.5) {\(q_2\)};
    \node[state] (Q3) at (3, -1.5) {\(q_3\)};
  

    \path (Q1) edge [] node
              {\(\braces{a, b}\)} (Q2);
    \path (Q2) edge [loop right] node
              {\(\braces{a}\)} (Q2);
    \path (Q2) edge [] node
              {\(\braces{b}\)} (Q3);
    \path (Q3) edge [loop right] node
              {\(\braces{a, c}\)} (Q3);
    \path (Q3) edge [] node
              {\(\braces{c}\)} (Q1);
  \end{tikzpicture}
  \end{center}
  There are a few things to note in our model:
  \begin{enumerate}
    \item[(1)]
      The automata is non-deterministic, because from \(q_3\),
      there are multiple transitions that may be taken.

    \item[(2)]
      We lack the notion of a start and final state.
      Rather, every state is treated as both start and final in
      this perspective to make embedding slightly easier.
  \end{enumerate}
  To embed this into our model in several steps.
  First, take \(M\) to be the monoid generated by \(\Sigma\),
  where monoid multiplication is taken to be string concatenation,
  and unit \(1\) is aliased as \(\varepsilon\) the empty string.
  \begin{align*}
    M = \parens{\Sigma, \cdot, 1}
  \end{align*}
  The \(M\)-semiring is generated using the powersets of \(M\),
  where the unit of addition \(0\) is equivalent to the
  empty set \(\emptyset\):
  \begin{align*}
    R = \parens{\powset{M}, \cup, \cdot, 0, 1}
  \end{align*}
  To demonstrate a better picture of how this works, we construct a
  transition matrix \(A\) for the automata
  that acts on the \(\parens{M, 3}\)-semimodule.
  Here we have \(3\) because there are \(3\) states.
  Let \(A_{i, j}\) denote the transition set from state \(i\) to
  state \(j\).
  \begin{align*}
    A =
      \begin{bmatrix}
        0 & \braces{a, b} & 0 \\
        0 & \braces{a} & \braces{b} \\
        \braces{c} & 0 & \braces{a, c}
      \end{bmatrix}
  \end{align*}
  In order to perform string concatenation towards the right,
  transition matrices act by right-matrix multiplication.
  That is, if \(v \in R^3\) is the initial (row) vector,
  then the subsequent state is \(v A\).

  To briefly demonstrate, two transitions of the matrix \(A\) appears as
  follows:
  \begin{align*}
    A^2 =
      \begin{bmatrix}
        0 & \braces{a, b} & 0 \\
        0 & \braces{a} & \braces{b} \\
        \braces{c} & 0 & \braces{a, c}
      \end{bmatrix} 
      \begin{bmatrix}
        0 & \braces{a, b} & 0 \\
        0 & \braces{a} & \braces{b} \\
        \braces{c} & 0 & \braces{a, c}
      \end{bmatrix}
      =
      \begin{bmatrix}
        0 & \braces{aa, ba} & \braces{ab, bb} \\
        \braces{bc} & \braces{aa} & \braces{ab, ba, bc} \\
        \braces{ac, cc} & \braces{ca, cb} & \braces{aa, ac, ca, cc}
      \end{bmatrix}
  \end{align*}
  In general, from graph theory, \(A^k\) denotes the \(k\)th consecutive
  transition using \(A\),
  and each entry \(A^k _{i, j}\) is the set of strings that will
  get from state \(q_i\) to \(q_j\) in \(k\) steps.

\end{example}

