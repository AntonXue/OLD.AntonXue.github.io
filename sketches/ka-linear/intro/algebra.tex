\subsection{Algebraic Structures}

\begin{definition}[Monoid]
  A monoid is a set \(S\) with a binary operation
  \(\type{\circ}{S \times S}{S}\) (multiplication) with the following axioms:
  \begin{enumerate}
    \item[(1)]
      \(\circ\) is associative.

    \item[(2)]
      There exists an identity element \(e \in S\) such that:
      \begin{align*}
        \forall a \in S \st e \circ a = a = a \circ e
      \end{align*}
  \end{enumerate}
\end{definition}

\begin{definition}[Commutative Monoid]
  A monoid \(\parens{S, \circ}\) is commutative if \(\circ\) is commutative.
\end{definition}

\begin{definition}[Semiring]
  A semiring is a set \(R\) with two binary operations
  \(\type{+}{R \times R}{R}\) (addition) and
  \(\type{\cdot}{R \times R}{R}\) (multiplication) with the following axioms:
  \begin{enumerate}
    \item[(1)]
      \(\parens{R, +}\) is a commutative monoid with identity \(0\).

    \item[(2)]
      \(\parens{R, \cdot}\) is a monoid with identity \(1\).

    \item[(3)]
      Multiplication left and right distributes over addition.

    \item[(4)]
      Multiplication by \(0\) annihilates \(R\).
  \end{enumerate}
\end{definition}


\begin{definition}[Left Module Over a Semiring]
  A left module \(M\) over a semiring \(R\)
  is a set with two binary operations
  \(\type{+}{M \times M}{M}\) (addition)
  and \(\type{\cdot}{R \times M}{M}\) (scalar multiplication)
  with the following axioms:
  \begin{enumerate}
    \item[(1)]
      \(\parens{M, +}\) is an abelian group.

    \item[(2)]
      For all \(s, r \in R\) and \(x ,y \in M\):
      \begin{align*}
        r \cdot \parens{x + y}
          &= \parens{r \cdot x} + \parens{r \cdot y} \\
        \parens{r + s} \cdot x
          &= \parens{r \cdot x} + \parens{s \cdot x} \\
        \parens{r \cdot s} \cdot x
          &= r \cdot \parens{s \cdot x} \\
        1_{R} \cdot x
          &= x
      \end{align*}
  \end{enumerate}
\end{definition}


\begin{definition}[Right Module Over a Semiring]
  A right module would be defined similarly, except
  all instances of scalar multiplication happen on the right side.
\end{definition}


\begin{definition}[Bimodule Over a Semiring]
  A bimodule is a module that is both a left module and a right module
  with respect to scalar multiplication.
\end{definition}

If \(M\) is a module over a semiring \(R\),
we write \(M / R\) for shorthand.


\begin{definition}[\(n\)-length Module]
  An \(n\)-length module \(M / R\)
  consists of \(n\) elements:
  \begin{align*}
    M = \parens{r_1, \ldots, r_n}
    \qquad
    \qquad
    r_1, \ldots, r_n \in R
  \end{align*}
  
\end{definition}


\begin{definition}[Linear Transformation]
  A linear transformation is a homomorphic map
  \(\type{T}{V}{W}\) between two modules \(V / R\) and \(W / R\)
  that preserves addition and scalar multiplication.
  That is,
  for all \(r \in R\) and \(x, y \in V\):
  \begin{align*}
    T\parens{0_R \cdot x}
      &= 0_W \\
    T\parens{r\cdot x + y}
      &= r\cdot T\parens{x} + T\parens{y}
  \end{align*}
\end{definition}

We may use matrices to represent linear transformations
between \(n\)-length modules.
Let \(M_{n \times n} \parens{R}\)
be the set of \(n \times n\) dimensional
matrices with entries from semiring \(R\).

\begin{definition}[Matrix Module (Left) Multiplication]
If \(A \in M_{n \times n}\parens{R}\),
and \(M / R\) is some \(n\)-length module.
Define left matrix-module multiplication as follows:
\begin{align*}
  \parens{AM}_{i}
    = \sum_{k = 1}^{n} A_{i, k} M_k
\end{align*}
\end{definition}

Here the result \(AM\) is another \(n\)-length module over semiring \(R\).
In particular, this action corresponds to left multiplication by the matrix.
If instead we wish to define right multiplication:

\begin{definition}[Matrix Module (Right) Multiplication]
If \(A \in M_{n \times n}\parens{R}\),
and \(M / R\) is some \(n\)-length module.
Define right matrix-module multiplication as follows:
\begin{align*}
  \parens{MA}_{i}
    = \sum_{k = 1}^{n} M_k A_{k, i}
\end{align*}
\end{definition}

Typically we take the module \(M\) to have a column vector representation
during left multiplication,
and a row vector representation for right multiplication.

