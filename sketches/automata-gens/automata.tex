\documentclass[12pt]{article}

% Packages
\usepackage[margin=5em]{geometry} % 1 cm = 2.84528 em
\usepackage[backend=bibtex]{biblatex}
\bibliography{sources}
% \nocite{*}

\usepackage{lipsum}

% Paragraphs
\setlength{\parindent}{0em}
\setlength{\parskip}{1em}

% Includes
% Includes
\usepackage{amsfonts}
\usepackage{amsmath}
\usepackage{amssymb}
\usepackage{amsthm}
\usepackage{comment}
\usepackage[colorlinks]{hyperref}
\usepackage{letltxmacro}
\usepackage{listings}
\usepackage{stmaryrd}

% Spacing
\let\uspace\undefined
\newcommand{\uspace}{\ensuremath{\ }}

% Numbers
\let\N\undefined
\newcommand{\N}{\ensuremath{\mathbb{N}}}

\let\Z\undefined
\newcommand{\Z}{\ensuremath{\mathbb{Z}}}

\let\Zz\undefined
\newcommand{\Zz}{\ensuremath{\mathbb{Z}^{\geq 0}}}

\let\Zp\undefined
\newcommand{\Zp}{\ensuremath{\mathbb{Z}^{+}}}

\let\Q\undefined
\newcommand{\Q}{\ensuremath{\mathbb{Q}}}

\let\R\undefined
\newcommand{\R}{\ensuremath{\mathbb{R}}}

\let\Rz\undefined
\newcommand{\Rz}{\ensuremath{\mathbb{R}^{\geq 0}}}

\let\Rp\undefined
\newcommand{\Rp}{\ensuremath{\mathbb{R}^{+}}}

\let\C\undefined
\newcommand{\C}{\ensuremath{\mathbb{C}}}

\let\mcal\undefined
\newcommand{\mcal}[1]{\ensuremath{\mathcal{#1}}}

\let\mbb\undefined
\newcommand{\mbb}[1]{\ensuremath{\mathbb{#1}}}

% Grouping
\let\parens\undefined
\newcommand{\parens}[1]{\ensuremath{\left(#1\right)}}

\let\brackets\undefined
\newcommand{\brackets}[1]{\ensuremath{\left[#1\right]}}

\let\braces\undefined
\newcommand{\braces}[1]{\ensuremath{\left\{#1\right\}}}

\let\angles\undefined
\newcommand{\angles}[1]{\ensuremath{\left\langle#1\right\rangle}}

\let\ceil\undefined
\newcommand{\ceil}[1]{\ensuremath{\left\lceil#1\right\rceil}}

\let\floor\undefined
\newcommand{\floor}[1]{\ensuremath{\left\lfloor#1\right\rfloor}}

% Sets and Function Spaces
\let\type\undefined
\newcommand{\type}[3]{\ensuremath{#1 \colon #2 \to #3}}

\let\powset\undefined
\newcommand{\powset}[1]{\ensuremath{2^{#1}}}

\let\mod\undefined
\newcommand{\mod}{\ensuremath{\uspace\mathrm{mod}\uspace}}

\let\ker\undefined
\newcommand{\ker}{\ensuremath{\mathrm{ker}}}

\let\dom\undefined
\newcommand{\dom}{\ensuremath{\mathrm{dom}}}

\let\ran\undefined
\newcommand{\ran}{\ensuremath{\mathrm{ran}}}

\let\im\undefined
\newcommand{\im}{\ensuremath{\mathrm{im}}}

\let\coker\undefined
\newcommand{\coker}{\ensuremath{\mathrm{coker}}}

\let\codim\undefined
\newcommand{\codim}{\ensuremath{\mathrm{codim}}}

% Complex Numbers
\let\Re\undefined
\newcommand{\Re}{\ensuremath{\mathrm{Re}}}

\let\Im\undefined
\newcommand{\Im}{\ensuremath{\mathrm{Im}}}

% Probability
\let\Pr\undefined
\newcommand{\Pr}{\ensuremath{\mathrm{Pr}}}

\let\E\undefined
\newcommand{\E}{\ensuremath{\mathrm{E}}}

\let\Var\undefined
\newcommand{\Var}{\ensuremath{\mathrm{Var}}}

\let\Cov\undefined
\newcommand{\Cov}{\ensuremath{\mathrm{Cov}}}

% Integrals
\let\dee\undefined
\newcommand{\dee}[1]{\ensuremath{\uspace d #1}}

% Linear Algebra
\let\abs\undefined
\newcommand{\abs}[1]{\ensuremath{\left\lvert#1\right\rvert}}

\let\norm\undefined
\newcommand{\norm}[1]{\ensuremath{\left\lVert#1\right\rVert}}


% Computational Complexity
\let\class\undefined % Complexity class
\newcommand{\class}[1]{\ensuremath{\mathbf{#1}}}

\let\prob\undefined % Complexity problem
\newcommand{\prob}[1]{\ensuremath{\text{#1}}}

% Theorem environment
\theoremstyle{plain}
\newtheorem{theorem}{Theorem}

\theoremstyle{plain}
\newtheorem{lemma}{Lemma}

\theoremstyle{plain}
\newtheorem{claim}{Claim}

\theoremstyle{plain}
\newtheorem{fact}{Fact}

\theoremstyle{plain}
\newtheorem{remark}{Remark}

\theoremstyle{plain}
\newtheorem{definition}{Definition}

\theoremstyle{plain}
\newtheorem{example}{Example}

\theoremstyle{plain}
\newtheorem{question}{Question}

% Code environment
\lstdefinestyle{plainsty}{
  basicstyle=\small\ttfamily,
  language=C,
  xleftmargin=\parindent,
  aboveskip=1em,
  belowskip=1em,
  showspaces=false,
  showstringspaces=false,
  keywordstyle = {},
}

\lstnewenvironment{pcode}{\lstset{style=plainsty}}{}

\let\pinl\undefined
\newcommand{\pinl}{\lstinline[style=plainsty]}

\newcommand*{\SavedLstInline}{} % Allows plain code usage in math mode.
\LetLtxMacro\SavedLstInline\pinl
\DeclareRobustCommand*{\pinl}{%
  \ifmmode
    \let\SavedBGroup\bgroup
    \def\bgroup{%
      \let\bgroup\SavedBGroup
      \hbox\bgroup
    }%
  \fi
\SavedLstInline}

\let\ttcode\undefined
\newcommand{\ttcode}[1]{\small{\texttt{#1}}}


% Text markings

\let\tturl\undefined
\newcommand{\tturl}[1]{\href{#1}{\texttt{#1}}}

\let\red\undefined
\newcommand{\red}[1]{\textbf{\color{red}#1}}


% Abbreviations
\let\st\undefined
\newcommand{\st}{\ensuremath{\uspace\colon\uspace}}

\let\ow\undefined
\newcommand{\ow}{\ensuremath{\text{otherwise}}}


% \input{aut-lib.tex}


% Author
\title{Automata and Some Generalizations}
% \author{Anton Xue}
% \date{\today}
\date{}

% Document
\begin{document}
\maketitle


\subsection{Deterministic Finite Automata}
A deterministic finite automata is often presented as a quintuple:
\begin{align*}
  \mcal{A}
    = \parens{\Sigma, Q, \delta, s, F}
\end{align*}
Where \(\Sigma\) is a finite alphabet,
\(Q\) is a finite set of states,
\(\type{\delta}{\Sigma \times Q}{Q}\)
is the transition function,
\(s \in Q\) is a special initial state,
and \(F \subseteq Q\) is the set of final states.

One particular view may see \(\delta\) as a matrix \(M\),
where the entry \(M_{i, j} \subseteq \Sigma\)
is the letters of \(\Sigma\)
that takes state \(q_i \in Q\) to \(q_j \in Q\).


\subsection{Non-deterministic Finite Automata}
A non-deterministic finite automata may also be presented as a quintuple:
\begin{align*}
  \mcal{A} = \parens{\Sigma, Q, \Delta, s, F}
\end{align*}
As with before, \(\Sigma\) is a finite alphabet,
\(Q\) is a finite set of states, \(s \in Q\) is a special start
state, and \(F \subseteq Q\) is the set of final states.
What is different here from deterministic finite automata,
is that the transition function
\(\type{\Delta}{\Sigma \times Q}{\powset{Q}}\) maps into a set of states
rather than a single state.
This represents the ``non-deterministic'' nature of a transition,
where a single letter read may induce multiple possible states.

As with before, \(\Delta\) may be seen as a matrix \(M\).
The presentation is the same as that of a deterministic finite
automata's, where \(M_{i, j} \subseteq \Sigma\) is the letters of \(\Sigma\)
that take \(q_i \in Q\) to \(q_j \in Q\).


\subsection{Semiring Automata}
The semiring automata is a formalization of a
non-deterministic finite automata using algebraic structures.
Take the semiring automata to be a quintuple:
\begin{align*}
  \mcal{A} = \parens{\Sigma, Q, \Delta, s, F}
\end{align*}
Like before, \(\Sigma\) is a finite alphabet that generates
the free semiring \(R\),
\(Q\) is a countable set of states,
\(\type{\Delta}{R \times Q}{\powset{Q}}\) is the transition function,
\(s \in Q\) is the special start state,
and \(F \subseteq Q\) is the set of final states.

Similar to before,
if \(Q\) is finite,
the matrix representation \(M\) of the transition
function \(\Delta\) has entry \(M_{i, j} \in R\) as the
elements that map state \(q_i \in Q\) to \(q_j \in Q\).
Note that we now permit each entry of the transition matrix to be
an arbitrary element of \(R\) rather than a set of single letters of \(\Sigma\).


\subsection{Division Semiring Automata}
The division semiring automata introduces the notion of ``negative'' words
through multiplicative inverses.
Similar to the semiring automata,
we take the division semiring automata as the quintuple:
\begin{align*}
  \mcal{A} = \parens{\Sigma, Q, \Delta, s, F}
\end{align*}
Here, \(\Sigma\) is a finite alphabet that generates the
free division semiring \(R\),
\(Q\) is a countable set of states,
\(\type{\Delta}{R \times Q}{\powset{Q}}\) is the transition function,
\(s \in Q\) is the special start state,
and \(F \subseteq Q\) is the set of final states.

If \(Q\) is finite,
let \(M\) be the matrix representation of the transition function \(\Delta\),
where \(M_{i, j} \in R\) maps state \(q_i \in Q\) to \(q_j \in Q\).





\printbibliography

\end{document}

