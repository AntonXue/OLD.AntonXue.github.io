
\begin{abstract}
Often an approximation is sufficient.
It may be the case that a complex system can be well-approximated
by the behavior of a much smaller and simpler proxy.
A good proxy is one in which the representation is small
yet the error of approximation can be rigorously bounded;
it allows us to easily study and understand complex behavior
at a simplified model.

In this work we study how regular languages can be embedded
into metric spaces.
This embedding allows us to rigorously discuss what it means for
two regular languages to be similar:
exactly what it means for them to share many strings and differ on few.
A good solution to this problem allows us to introduce
novel approximation techniques to areas such as
language learning, program synthesis,
and software verification.
For instance, achieving results in learning, synthesis, and verification
that are not exact, but are good enough,
may still be invaluable.

Our work explores this problem in the context of measure theory,
algebraic formalization, and language separation.
We demonstrate how the space of languages can be embedded into a
measure space, from which a metric space can be induced
under technical conditions.
Additionally, we give an algebraic formulation of regular languages
and non-deterministic finite automata, with an attention to
linear spaces over semirings.
Furthermore, we describe how to use boolean satisfiability to compute
a minimal non-deterministic finite automata that can
be used to classify two disjoint sets of strings.
Finally, we present graphical visualization techniques
for sets of strings that will benefit future research endeavors,
with relevant code attached in the appendix.



\end{abstract}

