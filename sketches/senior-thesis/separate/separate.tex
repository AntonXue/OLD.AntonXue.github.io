
\section{Separating Automata}

There are several ways to define a metric over the space of strings.
For instance,
the edit distance between two strings defines a metric.

Another metric that can be defined over two strings is with respect to an
automata.
For strings \(w\) and \(v\),
let \(\mcal{A}\) be the smallest automata that accepts \(w\) and rejects
\(v\).
The distance can then be defined as follows:
\begin{align*}
  d\parens{w, v} = \frac{1}{2^{\abs{\mcal{A}}}}
\end{align*}
Where \(\abs{A}\) here denotes the number of states of \(\mcal{A}\).

\red{INTUITION OF WHAT THIS METIRC MEANS}


The question is then how to promote this to compare over two sets of strings
rather than just two strings itself.
That is, given two sets of strings \(P\) and \(N\),
where we call \(P\) the set of positive strings and \(N\) the set of
negative strings,
let \(\mcal{A}\) be the smallest NFA that accepts all the strings in \(P\)
and rejects all the strings in \(N\),
where small refers to the number of states.
can a metric be defined with respect to \(\mcal{A}\)?
We would like to make a statement like:
\begin{align*}
  d\parens{P, N} = \frac{1}{2^{\abs{\mcal{A}}}}
\end{align*}
However it's not immediately clear that this forms a metric,
or if it does at all.
Nevertheless, the method of finding a minimal separating NFA is
still of theoretical interest.

We achieve this by constructing a boolean satisfiability formula
that encodes \(P\) and \(N\),
and is satisfiable if and only if such \(\mcal{A}\) exists.
There are several high-level insights that we leverage:
\begin{enumerate}
  \item[(1)]
    NFAs can be represented as directed multi-edge graphs
    where each edge is labeled by one letter from \(\Sigma\).
    Self-loops are permitted here.
    In other words, let \(e_{i, j, \sigma}\) be an indicator
    variable encodes the indicator of a transition from state \(q_i\)
    to state \(q_j\) on the letter \(\sigma\).

  \item[(2)]
    For each \(u \in P\),
    we can create a formula that forces a sequence of edge walks
    resulting in a final state in \(\mcal{A}\).
    Similarly for each \(v \in N\) we can encode a sequence that will force
    a rejection of \(v\) in \(\mcal{A}\).

\end{enumerate}

We first construct a formula that will force \(\mcal{A}\) to accept
a word \(w\) if and only if the formula is satisfied.
Let \(y_{i, t}^w\) denote be an indicator variable
to show that \(\mcal{A}\) is at state \(q_i\) at time \(t\),
with \(1 \leq i \leq n\) and \(1 \leq t \leq \abs{w} + 1\).
Note that since each letter of \(w\) acts as a transition,
the automata will occupy \(\abs{w} + 1\) possibly repeated
states during its accepting run.
Then:
\begin{align*}
  \rho_w \equiv
    \bigwedge_{1 \leq t \leq \abs{w} + 1}
      \brackets{\bigwedge_{1 \leq i, j \leq n}
        \neg \parens{y_{i, t} ^w \land y_{j, t} ^w}}
\end{align*}
Forces the automata to be in only one state at any given time \(t\)
while reading \(w\).
To accompany this, let \(e_{i, j, \sigma}\) to denote
that \(\mcal{A}\) has a transition edge from \(q_i\) to \(q_j\)
on letter \(\sigma\).
Similarly:
\begin{align*}
  \pi_w \equiv
    \bigwedge_{1 \leq t \leq \abs{w}}
      \brackets{\bigvee_{1 \leq i, j \leq n}
      \parens{y_{i, t} ^w \land y_{j, t + 1} ^w \land e_{i, j, w_t}}}
\end{align*}
Additionally, we can force boundary conditions to ensure that \(\mcal{A}\)
begins reading \(w\) on a starting state and ends on an accepting state:
\begin{align*}
  \gamma_{w} \equiv
    \brackets{\bigvee_{1 \leq i, j \leq n}
        \parens{e_{i, j, w_{1}} \land s_i}}
      \lor
    \brackets{\bigvee_{1 \leq i, j \leq n}
        \parens{e_{i, j, w_{\abs{w}} \land f_j}}}
\end{align*}
Where \(s_i\) and \(f_j\) are indicator variables to express that
\(q_i\) is a starting state and \(q_j\) is a final state respectively.
Then finally we set:
\begin{align*}
  \varphi_{w} \equiv \rho_w \land \pi_w \land \gamma_w
\end{align*}
Then \(\mcal{A}\) will only accept \(w\) if and only if \(\varphi_w\)
is satisfiable.
Finally:
\begin{align*}
  \Phi_{P, N} \equiv
    \brackets{\bigwedge_{u \in P} \varphi_u}
      \land
    \brackets{\bigwedge_{v \in N} \neg \varphi_v}
\end{align*}
By construction, \(\Phi_{P, N}\) is true if and only if \(\mcal{A}\)
accepts all \(P\) and rejects all \(N\).

Suppose that \(\Sigma\) is known.
If \(\Phi_{P, N}\) is satisfiable,
then \(\mcal{A} = \parens{\Sigma, Q, \Delta, S, F}\) can be extracted as:
\begin{align*}
  Q &= \braces{q_1, \ldots, q_n} \\
  \Delta &= \braces{\parens{\parens{q_i, \sigma}, q_j} \st e_{i, j, \sigma} = \top} \\
  S &= \braces{q_i \st s_i = \top} \\
  F &= \braces{q_j \st f_j = \top} \\
\end{align*}



