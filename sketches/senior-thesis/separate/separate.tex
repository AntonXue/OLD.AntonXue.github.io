
\section{Separating Automata}

There are several ways to define a metric over the space of strings.
For instance,
the edit distance between two strings defines a metric.

Another metric that can be defined over two strings is with respect to an
automata.
For strings \(w\) and \(v\),
let \(\mcal{A}\) be the smallest automata that accepts \(w\) and rejects
\(v\).
The distance can then be defined as follows:
\begin{align*}
  d\parens{w, v} = \frac{1}{2^{\abs{\mcal{A}}}}
\end{align*}
Where \(\abs{A}\) here denotes the number of states of \(\mcal{A}\).
Intuitively, the larger an automata is required to distinguish two strings,
the closer they are.


This seems like a fairly natural metric,
so the question is then how to promote this to
compare over two sets of strings rather than just two strings itself.
Since languages are no more than sets of strings,
this upgrade will yield a metric over languages rather than over strings.

A simple way to upgrade a metric over a set to its power set
(in this context from between strings to between languages)
is the Hausdorff metric.
In general, given any metric space \(\parens{M, d}\),
the Hausdorff metric space for this space is \(\parens{\powset{M}, d_H}\)
given as:
\begin{align*}
  d_H\parens{X, Y} =
    \max\braces{\sup_{x \in X} \inf_{y \in Y} d\parens{x, y},
                \sup_{y \in Y} \inf_{x \in X} d\parens{x, y}}
\end{align*}
In other words, between two sets \(X, Y \subseteq M\),
the Hausdorff distance \(d_H\) between them is the distance of the most
extreme pair of points \(x \in X\) and \(y \in Y\)
with respect to the original metric \(d\).

However we hope to do better.
NFAs can be used to separate strings, but they can also be used to separate
two sets of strings.
That is, given two disjoint sets of strings \(P\) and \(N\),
where we call \(P\) the set of positive strings and \(N\) the set of
negative strings,
let \(\mcal{A}\) be the smallest NFA that accepts all the strings in \(P\)
and rejects all the strings in \(N\),
where small refers to the number of states.
Can a metric be defined with respect to \(\mcal{A}\)?
We would like to make a statement like:
\begin{align*}
  d\parens{P, N} = \frac{1}{2^{\abs{\mcal{A}}}}
\end{align*}
However it's not immediately clear that this forms a metric,
or if it does at all.
Nevertheless, the method of finding a minimal separating NFA is
still of theoretical interest.
This is type of problem is known to be
\class{NP-Complete}~\cite{gold1978complexity},
but having a clear embedding and decision procedure is still useful
for exploring feasibility.


We achieve this by constructing a boolean satisfiability formula
that encodes \(P\) and \(N\),
and is satisfiable if and only if such \(\mcal{A}\) exists.
Our embedding is inspired by techniques from~\cite{neider2018learning}.
There are several high-level insights that we leverage:
\begin{enumerate}
  \item[(1)]
    NFAs can be represented as directed multi-edge graphs
    where each edge is labeled by one letter from \(\Sigma\).
    Self-loops are permitted here.
    In other words, let \(e_{i, j, \sigma}\) be an indicator
    variable encodes the indicator of a transition from state \(q_i\)
    to state \(q_j\) on the letter \(\sigma\).

  \item[(2)]
    For each \(u \in P\),
    we can create a formula that forces a sequence of edge walks
    resulting in a final state in \(\mcal{A}\).
    Similarly for each \(v \in N\) we can encode a sequence that will force
    a rejection of \(v\) in \(\mcal{A}\).

\end{enumerate}

We first construct a formula that will force \(\mcal{A}\) to accept
a word \(w\) if and only if the formula is satisfied.
Let \(y_{i, t}^w\) denote be an indicator variable
to show that \(\mcal{A}\) is at state \(q_i\) at time \(t\),
with \(1 \leq i \leq n\) and \(1 \leq t \leq \abs{w} + 1\).
Note that since each letter of \(w\) acts as a transition,
the automata will occupy \(\abs{w} + 1\) possibly repeated
states during its accepting run.
Then:
\begin{align*}
  \rho_w \equiv
    \bigwedge_{1 \leq t \leq \abs{w} + 1}
      \brackets{\bigwedge_{1 \leq i, j \leq n}
        \neg \parens{y_{i, t} ^w \land y_{j, t} ^w}}
\end{align*}
Forces the automata to be in only one state at any given time \(t\)
while reading \(w\).
To accompany this, let \(e_{i, j, \sigma}\) to denote
that \(\mcal{A}\) has a transition edge from \(q_i\) to \(q_j\)
on letter \(\sigma\).
Similarly:
\begin{align*}
  \pi_w \equiv
    \bigwedge_{1 \leq t \leq \abs{w}}
      \brackets{\bigvee_{1 \leq i, j \leq n}
      \parens{y_{i, t} ^w \land y_{j, t + 1} ^w \land e_{i, j, w_t}}}
\end{align*}
Additionally, we can force boundary conditions to ensure that \(\mcal{A}\)
begins reading \(w\) on a starting state and ends on an accepting state:
\begin{align*}
  \gamma_{w} \equiv
    \brackets{\bigvee_{1 \leq i, j \leq n}
        \parens{e_{i, j, w_{1}} \land s_i}}
      \land
    \brackets{\bigvee_{1 \leq i, j \leq n}
        \parens{e_{i, j, w_{\abs{w}}} \land f_j}}
\end{align*}
Where \(s_i\) and \(f_j\) are indicator variables to express that
\(q_i\) is a starting state and \(q_j\) is a final state respectively.
Then finally we set:
\begin{align*}
  \varphi_{w} \equiv \rho_w \land \pi_w \land \gamma_w
\end{align*}
Then \(\mcal{A}\) will only accept \(w\) if and only if \(\varphi_w\)
is satisfiable.
Finally:
\begin{align*}
  \Phi_{P, N} \equiv
    \brackets{\bigwedge_{u \in P} \varphi_u}
      \land
    \brackets{\bigwedge_{v \in N} \neg \varphi_v}
\end{align*}
By construction, \(\Phi_{P, N}\) is true if and only if \(\mcal{A}\)
accepts all \(P\) and rejects all \(N\).

Suppose that \(\Sigma\) is known.
If \(\Phi_{P, N}\) is satisfiable,
then \(\mcal{A} = \parens{\Sigma, Q, \Delta, S, F}\) can be extracted as:
\begin{align*}
  Q &= \braces{q_1, \ldots, q_n} \\
  \Delta &=
    \braces{\parens{\parens{\sigma, q_i},
                    \braces{q_k \st e_{i, k, \sigma} = \top}}
            \st \exists j, \uspace e_{i, j, \sigma} = \top} \\
  S &= \braces{q_i \st s_i = \top} \\
  F &= \braces{q_j \st f_j = \top}
\end{align*}

In short, an automata is recovered based on the edge indicators
\(e_{i, j, \sigma}\),
each of which denotes one transition.

\begin{theorem}
  \(\varphi_w\) is satisfiable if and only if
  \(\mcal{A} = \parens{\Sigma, Q, \Delta, S, F}\) accepts \(w\).
\end{theorem}
\begin{proof}
  Consider a satisfiable configuration of \(\varphi_w\),
  which implies that \(\rho_w\), \(\pi_w\), and \(\gamma_w\)
  are all satisfied.
  From \(\rho_w\)
  there exists exactly one sequence of \(\parens{y_{i, t}^w}\)
  indicator variables that are each \(\top\):
  \begin{align*}
    y_{a, 1}^w, y_{b, 2}^w, \ldots, y_{z, \abs{w} + 1}^w
  \end{align*}
  Where \(1 \leq a, b, \ldots, z \leq n\) denote arbitrary state numbers.
  Furthermore, \(\pi_w\) being \(\top\) means that for each consecutive
  \(y_{i, t}^w\) and \(y_{j, t + 1}^w\) the edge indicator variable
  \(e_{i, j, w_{t}}\) is \(\top\).
  In essence, this together \(\rho_w\) and \(\pi_w\) then forces
  the existence of a path of transitions induced by \(w\).
  Additionally, the formula \(\gamma_w\) forces \(w\) to begin on an
  initial state and end on a final state.
  Thus, there is a path that makes \(\mcal{A}\) accept \(w\).

  For the converse, suppose that \(\mcal{A}\) accepts \(w\).
  This implies that in the graph representation of \(\mcal{A}\)
  there exists a path on which \(w\) is accepted.
  Consider the indicator variable \(e_{i, j, w_t}\) corresponding to each
  transition of \(w\).
  In \(\pi_w\) this means that on each such transition we can set
  \(y_{i, t}^w = \top\) and \(y_{i, t + 1}^w = \top\).
  This satisfies \(\pi_w\),
  and also satisfies \(\rho_w\) since we can just set all other
  \(y\) variables of \(w\) to \(\bot\).
  Furthermore, since \(\mcal{A}\) accepting \(w\) implies that
  \(w\) must begin on a starting state and end on a final state,
  this is automatic for \(\gamma_w\).
  Hence, \(\varphi_w\) is satisfiable.

\end{proof}

The consequence then extends to \(\Phi_{P, N}\),
which is a conjunction over all \(\varphi_w\) and \(\neg \varphi_w\)
sub-formulas.

\begin{theorem}
  \(\Phi_{P, N}\) is satisfiable if and only if
  \(\mcal{A} = \parens{\Sigma, Q, \Delta, S, F}\) accepts every string
  in \(P\) and rejects every string in \(N\).
\end{theorem}
\begin{proof}
  For each \(w \in P \cup N\),
  this is a consequence of \(\mcal{A}\) accepting \(w\)
  if and only if \(\varphi_w\) is satisfiable.
\end{proof}

Furthermore, this NFA will be of size \(n\) if
\(\Phi_{P, N}\) is satisfiable.
\begin{theorem}
  If \(\Phi_{P, N}\) is satisfiable, then there exists an NFA of size \(n\)
  that separates \(P\) and \(N\).
\end{theorem}
\begin{proof}
  The satisfiability of \(\Phi_{P, N}\) implies the existence of
  a graph defined over at most \(n\) vertices,
  each of which would correspond to a state of an NFA that separates
  \(P\) and \(N\).
\end{proof}

