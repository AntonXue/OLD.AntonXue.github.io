
\section{Measure Theoretic Approaches}

\red{MOTIVATIONAL TEXT HERE}


Consider a language \(L \subseteq \Sigma^\star\).
The \(n\)-splice of a language written as \(L^n\) is defined as:
\begin{align*}
  L^n = L \cap \Sigma^n
\end{align*}
We then have the following relations:
\begin{align*}
  L
    = \bigcup_{n = 0}^{\infty} L^n
    = \bigcup_{n = 0}^{\infty} \parens{L \cap \Sigma^n}
    = L \cap \bigcup_{n = 0}^{\infty} \Sigma^n
    = L \cap \Sigma^\star
    = L
\end{align*}
Suppose that \(\parens{\N, \powset{\N}, \eta}\) is a probabiliy measure space
on \(\N\) with the probability measure \(\eta\),
one way to define a measure \(\lambda_{\eta}\)
is as follows:
\begin{align*}
  \lambda_{\eta} \parens{L}
    = \sum_{n = 0}^{\infty}
        \frac{\abs{L^n}}{\abs{\Sigma^n}} \eta\parens{n}
\end{align*}

\begin{theorem}
  \(\parens{\Sigma^\star, \powset{\Sigma^\star}, \lambda_{\eta}}\)
  is a measure space.
\end{theorem}
\begin{proof}
  Since the \(\powset{\Sigma^\star}\) is the largest \(\sigma\)-algebra
  on \(\Sigma^\star\), it suffices to show that \(\lambda_{\eta}\) is
  a measure.

  To see that \(\emptyset\) is mapped to \(0\):
  \begin{align*}
    \lambda_{\eta}\parens{\emptyset}
      = \sum_{n = 0}^{\infty} \frac{\abs{\emptyset}}{\abs{\Sigma^n}} \eta\parens{n}
      = \sum_{n = 0}^{\infty} 0
      = 0
  \end{align*}
  Now take \(\parens{A_n} \subseteq \Sigma^\star\) to be a countable
  collection of disjoint sets.
  Write \(A_n ^k\) to denote the \(k\) splice of the \(n\)th set.
  In other words: \begin{align*}
    A_n = \bigcup_{k = 0}^{\infty} A_{n} ^k
  \end{align*}
  Observe that all such \(A_{n}^k\) are pairwise disjoint by construction,
  and so:
  \begin{align*}
    \lambda_{\eta} \parens{\bigcup_{n = 0}^{\infty} A_n}
      = \lambda_{\eta} \parens{\bigcup_{n = 0}^{\infty} \bigcup_{k = 0}^{\infty} A_n ^k}
      = \sum_{n = 0}^{\infty} \lambda_{\eta} \parens{\bigcup_{k = 0}^{\infty} A_n ^k}
      = \sum_{n = 0}^{\infty} \sum_{k = 0}^{\infty} \frac{\abs{A_n ^k}}{\abs{\Sigma^n}} \eta\parens{k}
      = \sum_{n = 0}^{\infty} \lambda_{\eta} \parens{n}
  \end{align*}
  We conclude that \(\parens{\Sigma^\star, \powset{\Sigma^\star}, \lambda_{\eta}}\)
  forms a measure space.

\end{proof}
We can generalize this more.
Suppose that \(\nu = \parens{\nu_n}\)
is a countable collection of measures where
each \(\nu_n\) is defined on the splice \(\Sigma^n\).
Then we can extend a definition of \(\lambda_{\eta, \nu}\) as:
\begin{align*}
  \lambda_{\eta, \nu} \parens{A}
    = \sum_{n = 0}^{\infty} \nu\parens{A^n} \eta\parens{n}
\end{align*}

\begin{theorem}
  \(\parens{\Sigma^\star, \powset{\Sigma^\star}, \lambda_{\eta, \nu}}\)
  is a measure space.
\end{theorem}
\begin{proof}
  As with before, we only show that \(\lambda_{\eta, \nu}\) is a measure.

  For \(\emptyset\) we have again:
  \begin{align*}
    \lambda_{\eta, \nu}\parens{\emptyset} = \sum_{n = 0}^{\infty} 0 = 0
  \end{align*}

  Again take \(\parens{A_n} \subseteq \Sigma^\star\) to be a
  countable disjoint collection of sets,
  and \(A_n ^k\) to be the \(k\) splice of \(A_n\).
  Then:
  \begin{align*}
    \lambda_{\eta, \nu} \parens{\bigcup_{n = 0}^{\infty} A_n}
      = \lambda_{\eta, \nu} \parens{\bigcup_{n = 0}^{\infty} \bigcup_{k = 0}^{\infty} A_n ^k}
      = \sum_{n = 0}^{\infty} \lambda_{\eta, \nu} \parens{\bigcup_{k = 0}^{\infty} A_n ^k}
      = \sum_{n = 0}^{\infty} \sum_{k = 0}^{\infty} \nu_k \parens{A_n ^k} \eta\parens{k}
      = \sum_{n = 0}^{\infty} \lambda_{\eta, \nu} \parens{A_n}
  \end{align*}
  This shows that
  \(\parens{\Sigma^\star, \powset{\Sigma^\star}, \lambda_{\eta, \nu}}\)
  is a measure space.
\end{proof}


