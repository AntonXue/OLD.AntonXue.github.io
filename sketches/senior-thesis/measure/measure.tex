
\section{Measure Theoretic Approaches}

Our initial efforts began from a simple question:
is it possible to assign a notion of size to a regular language?
We are interested in this perspective for several reasons.
First, because regular languages are closed under
intersection and set difference,
being able to quantiatively measure their intersection and set
difference would allow us to measure some type of similarity.
Second, being able to embed regular languages into a measure space,
as noted earlier,
would allow us to derive a metric function with respect to the
symmetric set difference.
The natural course of investigation leads to measure theory.

Consider a language \(L \subseteq \Sigma^\star\).
The \(n\)-splice of a language written as \(L^n\) is defined as:
\begin{align*}
  L^n = L \cap \Sigma^n
\end{align*}
We then have the following relations:
\begin{align*}
  L
    = \bigcup_{n = 0}^{\infty} L^n
    = \bigcup_{n = 0}^{\infty} \parens{L \cap \Sigma^n}
    = L \cap \bigcup_{n = 0}^{\infty} \Sigma^n
    = L \cap \Sigma^\star
    = L
\end{align*}
Suppose that \(\parens{\N, \powset{\N}, \eta}\) is a probabiliy measure space
on \(\N\) with the probability measure \(\eta\),
one way to define a measure \(\lambda_{\eta}\)
is as follows:
\begin{align*}
  \lambda_{\eta} \parens{L}
    = \sum_{n = 0}^{\infty}
        \frac{\abs{L^n}}{\abs{\Sigma^n}} \eta\parens{n}
\end{align*}

\begin{theorem}
  \(\parens{\Sigma^\star, \powset{\Sigma^\star}, \lambda_{\eta}}\)
  is a measure space.
\end{theorem}
\begin{proof}
  Since the \(\powset{\Sigma^\star}\) is the largest \(\sigma\)-algebra
  on \(\Sigma^\star\), it suffices to show that \(\lambda_{\eta}\) is
  a measure.

  To see that \(\emptyset\) is mapped to \(0\):
  \begin{align*}
    \lambda_{\eta}\parens{\emptyset}
      = \sum_{n = 0}^{\infty} \frac{\abs{\emptyset}}{\abs{\Sigma^n}} \eta\parens{n}
      = \sum_{n = 0}^{\infty} 0
      = 0
  \end{align*}
  Now take \(\parens{A_n} \subseteq \Sigma^\star\) to be a countable
  collection of disjoint sets.
  Write \(A_n ^k\) to denote the \(k\) splice of the \(n\)th set.
  In other words: \begin{align*}
    A_n = \bigcup_{k = 0}^{\infty} A_{n} ^k
  \end{align*}
  Observe that all such \(A_{n}^k\) are pairwise disjoint by construction,
  and so:
  \begin{align*}
    \lambda_{\eta} \parens{\bigcup_{n = 0}^{\infty} A_n}
      = \lambda_{\eta} \parens{\bigcup_{n = 0}^{\infty} \bigcup_{k = 0}^{\infty} A_n ^k}
      = \sum_{n = 0}^{\infty} \lambda_{\eta} \parens{\bigcup_{k = 0}^{\infty} A_n ^k}
      = \sum_{n = 0}^{\infty} \sum_{k = 0}^{\infty} \frac{\abs{A_n ^k}}{\abs{\Sigma^n}} \eta\parens{k}
      = \sum_{n = 0}^{\infty} \lambda_{\eta} \parens{n}
  \end{align*}
  We conclude that \(\parens{\Sigma^\star, \powset{\Sigma^\star}, \lambda_{\eta}}\)
  forms a measure space.

\end{proof}
We can generalize this more.
Suppose that \(\nu = \parens{\nu_n}\)
is a countable collection of measures where
each \(\nu_n\) is defined on the splice \(\Sigma^n\).
Then we can extend a definition of \(\lambda_{\eta, \nu}\) as:
\begin{align*}
  \lambda_{\eta, \nu} \parens{A}
    = \sum_{n = 0}^{\infty} \nu\parens{A^n} \eta\parens{n}
\end{align*}

\begin{theorem}
  \(\parens{\Sigma^\star, \powset{\Sigma^\star}, \lambda_{\eta, \nu}}\)
  is a measure space.
\end{theorem}
\begin{proof}
  As with before, we only show that \(\lambda_{\eta, \nu}\) is a measure.

  For \(\emptyset\) we have again:
  \begin{align*}
    \lambda_{\eta, \nu}\parens{\emptyset} = \sum_{n = 0}^{\infty} 0 = 0
  \end{align*}

  Again take \(\parens{A_n} \subseteq \Sigma^\star\) to be a
  countable disjoint collection of sets,
  and \(A_n ^k\) to be the \(k\) splice of \(A_n\).
  Then:
  \begin{align*}
    \lambda_{\eta, \nu} \parens{\bigcup_{n = 0}^{\infty} A_n}
      = \lambda_{\eta, \nu} \parens{\bigcup_{n = 0}^{\infty} \bigcup_{k = 0}^{\infty} A_n ^k}
      = \sum_{n = 0}^{\infty} \lambda_{\eta, \nu} \parens{\bigcup_{k = 0}^{\infty} A_n ^k}
      = \sum_{n = 0}^{\infty} \sum_{k = 0}^{\infty} \nu_k \parens{A_n ^k} \eta\parens{k}
      = \sum_{n = 0}^{\infty} \lambda_{\eta, \nu} \parens{A_n}
  \end{align*}
  This shows that
  \(\parens{\Sigma^\star, \powset{\Sigma^\star}, \lambda_{\eta, \nu}}\)
  is a measure space.
\end{proof}

Having defined a measure space corresponding to the space of languages,
we can leverage the earlier observation that measures induce a natural
metric through their symmetric set difference.

\begin{example}
  Fix an alphabet \(\Sigma\) and a probability measure \(\eta\) on
  the non-negative integers.
  For the measure space
  \(\parens{\Sigma^\star, \powset{\Sigma^\star}, \lambda_{\eta}}\) and
  languages \(L_1, L_2 \subseteq \Sigma^\star\),
  the distance between \(L_1\) and \(L_2\) can then be defined as:
  \begin{align*}
    d\parens{L_1, L_2} = \lambda_{\eta} \parens{L_1 \triangle L_2}
  \end{align*}

  Provided additional measures \(\parens{\nu_n}\) defined as above
  that extends to a measure space
  \(\parens{\Sigma^\star, \powset{\Sigma^\star}, \lambda_{\eta, \nu}}\),
  the definition then becomes:
  \begin{align*}
    d \parens{L_1, L_2} = \lambda_{\eta, \nu} \parens{L_1 \triangle L_2}
  \end{align*}

\end{example}


By embedding the set of strings into a measure space,
we therefore effectively extend all languages over \(\Sigma\)
into a metric space,
not just regular languages.
This offers a generalized framework,
but comes with several issues that must be addressed before
it becomes practical.

First, we did not specify the probability distributions to be used.
The specific distributions of interest would most likely be
problem-specific, and would require at least some justification.
When specifying this metric we had in mind a goemetric distribution,
which means that this metric would then favor shorter strings over longer ones.
That is, two languages \(L_1\) and \(L_2\) would be considered more similar
if they shared a large amount of shorter strings than longer ones.

Second, the summation here is infinite, meaning that a precise closed-form
solution is only possible in very special cases.
However because we probability distribution over the non-negative integers
is used, one may leverage an approximate metric.
That is, for a fixed probability distribution \(\nu\)
over the non-negative integers
and \(\varepsilon > 0\),
there exists \(N\) such that:
\begin{align*}
  \sum_{k = 0}^{N} \nu\parens{k} > 1 - \varepsilon
\end{align*}
By introducing such an \(\varepsilon\) error,
the sum becomes finite,
and can be refined to arbitrary precision supposing that
\(\nu\) is computable.



