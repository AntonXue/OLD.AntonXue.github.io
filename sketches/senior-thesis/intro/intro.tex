
\section{Introduction}

Language recognition is a fundamental problem in computer science.
Given an alphabet of unique symbols \(\Sigma\),
let \(\Sigma^\star\) denote the set of all possible finite strings over
the alphabet \(\Sigma\).
A language \(L \subseteq \Sigma^\star\) is nothing more than a set of strings,
and we use these terms interchangeably.
For some string \(w \in \Sigma^\star\),
we can then ask if \(w \in L\).
This is the language recognition problem,
and is no different than a question about set membership.

A number of problems in theoretical computer science can be formulated
in terms of language recognition.
Does this string belong to the set (language) of valid email addresses?
Does this stirng belong to the set (language) of valid
computer programs written in my favorite programming language?
Does this string belong to the set (language) of solutions
to an instance of the boolean satisfiability problem?

Of the many questions that can be formulated in terms of language recognition,
one of central questions pertains to recognition of regular languages.
Regular languages are simple yet powerful:
they are indispensible in their ability to model
finite state machines,
describe structural patterns in strings,
and behave as a simple programming language model.
Abstractly, regular languages are precisely the set of languages that
can be described by the class of regular expressions.
Equivalently, regular languages are the family of languages
that are recognized by (non)-deterministic finite automata.

However, because of the way they are defined, a ``flaw'' is that
two regular languages \(L_1\) and \(L_2\) may differ on a very small number
of strings, yet have very different representation sizes.
For two regular language \(L_1\) and \(L_2\) specified by,
for instance, regular expressions.
The set of strings shared \(L_1 \cap L_2\) may be a large fraction of both
languages, and we may wish to consider \(L_1\) and \(L_2\) to be similar.
On the other hand,
the (minimal) regular expression used to describe and represent
\(L_1\) may be very large, while the (minimal) regular
expression for \(L_2\) may be significantly smaller.
So how different is \(L_1\) from \(L_2\)?
In other words, how might we assign a metric between \(L_1\) and \(L_2\),
or more generally, over the space of regular languages?

One may attempt a classification with respect to their syntactic metric.
That is, a metric distance is defined with respect to a representation
without direct consideration for what strings the two sets may represent.
This would be a metric over the representation, be it regular
expressions, finite automata, or some other means.

Alternatively, one may seek to determine a metric in terms of
the sets of strings in \(L_1\) and \(L_2\), which would be a semantic metric.
This metric would be independent of representation,
and would be defined over the space of regular sets.

Ideally, we would like a pair of metrics on the space of regular languages,
one syntactic and one semantic,
that are highly correlated.
That is, the syntactic metric between \(L_1\) and \(L_2\) is small
if and only if the semantic metric is also small.
We set out to explore the different types of metrics for regular languages.

In this report,
we present our efforts and progress towards various metric spaces
on the space of (regular) languages.
Our efforts span exploration into measure theory,
rediscovering embedding of formal language theory into algebraic formulations,
using automata to separate sets,
and also interesting visualization techniques.



