
\section{Background}


\red{Filler text for background}


\subsection{Regular Languages and Finite Automata}

Language recognition is a fundamental problem in theoretical computer science.
Given an alphabet of unique symbols \(\Sigma\),
let \(\Sigma^\star\) denote the set of all possible finite strings over
the alphabet \(\Sigma\).
For some set of strings that we call a language \(L \subseteq \Sigma^\star\)
and string \(w \in \Sigma^\star\),
we then ask if \(w \in L\).
This is the language recognition problem.

A number of problems in theoretical computer science can be formulated
in terms of language recognition.
Does this string belong to the set (language) of valid email addresses?
Does this stirng belong to the set (language) of valid
computer programs written in my favorite programming language?
Does this string belong to the set (language) of solutions
to an instance of the boolean satisfiability problem?

Here we are primarily concerned with recognizing regular languages.
These are the languages that can be described by regular expressions,
and see widespread application in text parsing.
Equivalently stated, regular languages are precisely the set of languages
recognized by the set of non-deterministic finite automata (NFA),
which we aim to study here.

Formally, a NFA is a tuple
\(\parens{\Sigma, Q, \delta, S, F}\)
that represents a finite state transition machine
which accepts or rejects strings.
Here \(\Sigma\) is the alphabet,
\(Q\) is the set of states,
\(\type{\Delta}{\Sigma \times Q}{\powset{Q}}\) is the transition function,
\(S \subseteq Q\) is the set of initial states,
and \(F \subseteq Q\) is the set of final states.

A NFA accepts a string if there exists a sequence of transition
starting from some \(q_s \in S\)
that ends in \(q_f \in F\).
As the transition function maps to a set of possible states that may
be arbitrarily chosen,
only the existence of a transition sequence is necessary,
hence the term non-deterministic.

\begin{example}[NFA]
  \red{AAAAAAAAAAAAA}
\end{example}



\red{ALGEBRAIC FORMULATION?}






%%%%%%%%%%%%%%%%%%%%%%%%%%%%%%%%%%%%%%%%%%%%%%%%%%%%%%%%%%%%%%%%%%%%
\if false
A regular language is a language that can be recongnized by a regular
expression.
Such languages play a central role in theoretical comptuer science
and formal language theory due to their ability to simply describe a
large class of strings.
We now formalize these terms.

Given a finite set of unique symbols \(\Sigma\) called the alphabet,
write the set of all finite strings constructed over this alphabet
as \(\Sigma^\star\).
A string is nothing more than a finite sequences of symbols from an alphabet.
Let \(\varepsilon\) denote the empty string which contains no symbols.

\begin{example}
  Consider an alphabet \(\Sigma = \braces{a, \alpha, b, \beta, \div}\),
  examples of strings (finite sequences) that can be constructed
  with this alphabet include
  \begin{align*}
    ab\beta\beta\alpha
      \qquad \qquad
    \div\div a \beta
      \qquad \qquad
    a a a a a a a
      \qquad \qquad
    \varepsilon
      \qquad \qquad
  \end{align*}
\end{example}

A language \(L\) is nothing more than a set of strings.
In other words, \(L \subseteq \Sigma^\star\).
We are now ready to introduce the concept of a regular language.
Formally, regular languages are a family of languages inductively
defined as follows:

\begin{enumerate}
  \item[(1)]
    The empty set \(\emptyset\) and the empty string language
    \(\braces{\varepsilon}\) are regular languages.

  \item[(2)]
    For each symbol \(a \in \Sigma\),
    the singleton language \(\braces{a}\) is a regular language.

  \item[(3)]
    If \(A\) and \(B\) are regular languages,
    then so is their union \(A \cup B\),
    their concatenation \(A \cdot B\), and their Kleene star \(A^\star\),
    defined as:
    \begin{align*}
      A \cup B
        &= \braces{w \st w \in A \cup B} \\
      A \cdot B
        &= \braces{w_a \cdot w_b \st w_a \in A, w_b \in B} \\
      A^\star
        &= \braces{w^k \st k \in \N, w \in A}
    \end{align*}
    Where \(w^k\) is the k-fold concatenation of a string to itself,
    and \(w_a \cdot w_b\) is the concatenaiton of strings
    \(w_a\) and \(w_b\).
    Sometimes we write \(w_a w_b\) for concatenation when context is clear.

  \item[(4)]
    No other languages are regular.
\end{enumerate}

As alluded to earlier,
regular languages are precisely languages
that are recognized by the set of regular expressions.
For a regular expression \(R\) to recognize a language \(L\) means that
the regular expression \(R\) can describe every string in \(L\).
We now make this more concrete.
Given an alphabet \(\Sigma\),
a regular expression is defined inductively as follows


\begin{enumerate}
  \item[(1)]
    The empty regular expression denotes the empty language \(\emptyset\).
    The empty string \(\varepsilon\) describes the
    language containing only the empty string \(\braces{\varepsilon}\).
    For each symbol \(a \in \Sigma\), 
    the regular expression of just \(a\) matches the language
    \(\braces{a}\).

  \item[(2)]
    Given regular expressions \(R\) and \(S\),
    the following operations are regular expressions
    \begin{enumerate}
      \item[(a)]
        Concatenation \(RS\) is a regular expression that denotes
        the 

      \item[(b)]
      
      \item[(c)]
    \end{enumerate}

  \item[(3)]
\end{enumerate}

\fi
%%%%%%%%%%%%%%%%%%%%%%%%%%%%%%%%%%%%%%%%%%%%%%%%%%%%%%%%%%%%%%%%%%%%



\subsection{Metric Spaces}

The concept of a distance is formalized in mathematics through a metric space.
A metric space is a pair \(\parens{M, d}\) where \(M\) is a set
and \(\type{d}{M \times M}{\R}\) is known as the metric, or distance,
function that aims to assign a distance between any two members of \(M\).
A metric space comes equipped with the following axioms
that must hold for any \(x, y, z \in M\):
\begin{enumerate}
  \item[(1)]
    Non-negativity of \(d\): \(d\parens{x, y} \geq 0\).

  \item[(2)]
    Identity of indiscernibles:
    \(d\parens{x, y} = 0\) if and only if \(x = y\).

  \item[(3)]
    Symmetry:
    \(d\parens{x, y} = d\parens{y, x}\).

  \item[(4)]
    Triangle inequality:
    \(d\parens{x, z} \leq d\parens{x, y} + d\parens{y, z}\).

\end{enumerate}


\begin{example}
  \red{EXAMPLE OF METRIC SPACE HERE}
\end{example}


\subsection{Measure Theory}

\red{Filler text on measure theory}


\begin{example}
  \red{EXAMPLE 1 of MEASURE SPACE}
\end{example}



\begin{example}
  \red{EXAMPLE 2 of MEASURE SPACE}
\end{example}


