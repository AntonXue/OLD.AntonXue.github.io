\section{Pre-measures on Regular Languages}
We observe a few things.
First, regular languages are inherently finite in representation.
Furthermore, they are also closed under finite language union,
intersection, and complement.
This means that they form an algebra under these operations.
Therefore, it makes sense to begin by defining a pre-measure on
regular languages.

\subsection{Counting Pre-measure}
Let \(\premucount \colon \alphabet^\ast \rightarrow \brackets{0, \infty}\)
be the counting pre-measure defined for countable sets,
such that:
\begin{align*}
  \premucount\parens{\lang} = \abs{\lang}
\end{align*}

For a regular expression \(\regex\),
we wish to say something meaningful about the pre-measure of its
generated regular language
\(\premucount\parens{\reglang\parens{\regex}}\).
Then \(\premucount\) may be defined inductively in terms of
the regular expression as follows:
\begin{align*}
  \premucount\parens{\emptyset} = 0
    &\qquad \forall a \in \alphabet \st \premucount\parens{\braces{a}} = 1 \\
  \langmin\parens{\regex} = \regex_1 + \regex_2
    &\implies
      \premucount\parens{\reglang\parens{\regex}} =
        \premucount\parens{\reglang\parens{\regex_1}} +
        \premucount\parens{\reglang\parens{\regex_2}} \\
  \langmin\parens{\regex} = \regex_1 \cdot \regex_2
    &\implies
      \premucount\parens{\reglang\parens{\regex}} =
        \premucount\parens{\reglang\parens{\regex_1}} \cdot
        \premucount\parens{\reglang\parens{\regex_2}} \\
  \langmin\parens{\regex} = \regex_1 ^\ast
    &\implies
      \premucount\parens{\reglang\parens{\regex}}
        = \begin{cases}
            \infty & \quad \premucount\parens{\regex_1} > 0 \\
            0 & \quad \text{otherwise}
          \end{cases}
\end{align*}

Observe that almost by definition (and a combinatorial argument),
\(\premucount\parens{\reglang\parens{\regex}}\) counts the size of
the language corresponding to regular expression \(\regex\).

We now demonstrate that \(\premucount\) is indeed a pre-measure.

\begin{theorem}
  The function \(\premucount\) defines a pre-measure.
\end{theorem}
\begin{proof}
  Consider a countable alphabet \(\alphabet\) and the set of all
  regular expressions \(\mathcal{E}\).

  First observe that the empty language \(\emptyset\)
  is regular,
  and is generated by the empty string \(\empstr \in \mathcal{E}\).
  For this we have that \(\premucount\parens{\emptyset} = 0\),
  thereby satisfying the requirement that empty sets map to zero.
  
  Now, consider two regular languages
  \(\reglang\parens{\regex_1}\) and \(\reglang\parens{\regex_2}\)
  that are disjoint.
  Recall that the semantics of \(+\) means that:
  \begin{align*}
    \reglang\parens{\regex_1 + \regex_2} =
      \reglang\parens{\regex_1} \cup \reglang\parens{\regex_2}
  \end{align*}
  Furthermore, because \(\reglang\parens{\regex_1}\) and
  \(\reglang\parens{\regex_2}\) are disjoint, this means that
  \(\langmin\parens{\regex_1} + \langmin\parens{\regex_2}\) is in
  the desired canonical form.
  Hence, we expand this as follows:
  \begin{align*}
    \premucount
      \parens{\reglang\parens{\regex_1} \cup \reglang\parens{\regex_2}}
      = \premucount\parens{\reglang\parens{\regex_1 + \regex_2}}
      = \premucount\parens{\reglang\parens{\regex_1}} +
        \premucount\parens{\reglang\parens{\regex_2}}
  \end{align*}
  This satisfies the additivity rule for pre-measures.
\end{proof}



