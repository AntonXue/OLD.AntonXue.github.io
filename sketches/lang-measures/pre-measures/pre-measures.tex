\section{Pre-Measures on Regular Languages}
We observe a few things.
First, regular languages are inherently finite in representation.
Furthermore, they are also closed under finite language union,
intersection, and complement.
This means that they form an algebra under these operations.
Therefore, it makes sense to begin by defining a pre-measure on
regular languages.

\subsection{Counting Pre-measure}
Let \(\premucount\) be the counting pre-measure defined for countable sets,
such that:
\begin{align*}
  \premucount\parens{\lang} = \abs{\lang}
\end{align*}

For a regular expression \(\regex\),
we wish to say something meaningful about the pre-measure of its
generated regular language
\(\premucount\parens{\reglang\parens{\regex}}\).
Perhaps \(\premucount\) may be defined inductively in terms of
the regular expression as follows:
\begin{align*}
  \premucount\parens{\empstr} = 0
    &\qquad \forall a \in \alphabet \st \premucount\parens{a} = 1 \\
  \langmin\parens{E} = E_1 + E_2
    &\implies
      \premucount\parens{E} =
        \premucount\parens{E_1} + \premucount\parens{E_2} \\
  \langmin\parens{E} = E_1 \cdot E_2
    &\implies
      \premucount\parens{E} =
        \premucount\parens{E_1} \cdot \premucount\parens{E_2} \\
\end{align*}


