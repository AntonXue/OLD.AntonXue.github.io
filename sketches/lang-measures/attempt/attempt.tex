\section{Assigning Measures}




\if false

\subsection{Pre-measures on Regular Languages}
We first note that \(\parens{\alphabet, \reglang}\) is closed under
finite language union, finite language intersection,
and finite language complement.
This means that \(\parens{\alphabet, \reglang}\) forms an algebra
with respect to these operations.

We must be careful about extending \(\parens{\alphabet, \reglang}\) into a
\(\sigma\)-algebra, if we can even do so at all.
For instance, one of the canonical examples of a language that is not
regular, but is instead context-free, is the language the language
over \(\alphabet = \braces{a, b}\) such that
any sequence of \(a\)'s are followed by the same number of \(b\)'s:
\[
  \bigcup_{n \in \N} \braces{a^n b^n}
\]
Note that if the union were finite, the language would be regular,
but because it is instead countably infinite, it is not regular in this case.

\subsubsection{The Counting Measure}
Because regular languages are finite, we take the liberty to define
\(\langmin\parens{R}\) as the minimal reprsentation
Let us define a counting pre-measure \(\premucount\) first on \(R\)
as follows:
\begin{align*}
  \premucount\parens{\emptyset} &= 0 \\
  \premucount\parens{R_1 + R_2} &= 
\end{align*}


\fi

