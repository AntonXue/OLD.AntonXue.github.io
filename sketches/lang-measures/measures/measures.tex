\section{Measures on Languages}
Given a family of languages \(\mcal{L}\),
let \(\sigma\parens{\mcal{L}}\) be the \(\sigma\)-algebra generated on
\(\mcal{L}\) satisfying the following:

\begin{enumerate}
  \item[(1)]
    \begin{align*}
      \emptyset, \Sigma^\star \in \sigma\parens{\mcal{L}}
    \end{align*}

  \item[(2)]
    \begin{align*}
      L \in \sigma\parens{\mcal{L}}
        \implies
          L^c = \Sigma^\star \setminus L \in \sigma\parens{\mcal{L}}
    \end{align*}

  \item[(3)]
    \begin{align*}
      L_0, L_1, \ldots \in \sigma\parens{\mcal{L}}
        \implies
          \bigcup_{k = 0}^{\infty} L_k \in \sigma\parens{\mcal{L}}
    \end{align*}

\end{enumerate}

Then \(\parens{\mcal{L}, \sigma\parens{\mcal{L}}}\) is a measurable space.

\begin{remark}
If \(\mcal{L}\) happened to be a family of regular languages,
there is no guarantee that \(\sigma\parens{\mcal{L}}\)
will still be a family of regular languages.
A counter example is the following:
\begin{align*}
  L_0 = \braces{\varepsilon}
  \qquad
  L_1 = \braces{ab}
  \qquad
  L_2 = \braces{aabb}
  \qquad
  \ldots
  \qquad
  L_k = \braces{a^k b^k}
  \qquad
  \ldots
\end{align*}
But taking the countable union yields:
\begin{align*}
  \bigcup_{k = 0}^{\infty} L_k = \braces{a^k b^k \st k \in \Zz}
\end{align*}
Which is not regular.
\end{remark}


%%%%%%%%%%%%%%%%%%%%%%%%%%%%%%%%%%%%%%%%%%%%%%%%%%%%%%%%%%%%%%
\subsection{Measure 1: From Non-negative Integers}
We first consider the non-negative integers \(\Zz\).
Let \(\eta\) be a \(\sigma\)-finite measure on \(\Zz\).
The \(\sigma\)-finite conditions ensures that no strange singularities
occur for any integers under consideration.
We may later restrict \(\eta\) to be finite if necessary,
if we want nicer conditions.

Observe that, by abuse of notation:
\begin{align*}
  \Sigma^\star =
    \bigcup_{k = 0}^{\infty} \Sigma^k
\end{align*}
In English: \(\Sigma^\star\) is the union of the set (language) of finite
strings of length \(k\), denoted \(\Sigma^k\).

Because we assumed \(\abs{\Sigma} < \infty\), this also means that:
\(\abs{\Sigma^k} = \abs{\Sigma}^k\).

Consider now some language \(L \in \sigma\parens{\mcal{L}}\).
Also decompose \(L\) into disjoint sub-languages by length as follows,
with convenient subscripting:
\begin{align*}
  L = \bigcup_{k = 0}^{\infty} L_k
\end{align*}
Of course, \(L_k \subseteq \Sigma^k\).

Because we are able to precisely calculate \(\abs{\Sigma^k}\),
one ``natural'' way of defining a measure \(\lambda_{\eta}\) on
the measurable space \(\parens{\mcal{L}, \sigma\parens{\mcal{L}}}\)
is as follows:
\begin{align*}
  \lambda_{\eta} \parens{L}
    = \sum_{k = 0}^{\infty} \lambda_{\eta} \parens{L_k}
    = \sum_{k = 0}^{\infty} \frac{\abs{L_k}}{\abs{\Sigma^k}} \eta\parens{k}
\end{align*}

We claim that
\(\parens{\mcal{L}, \sigma\parens{\mcal{L}}, \lambda_{\eta}}\)
forms a measure space.

\begin{theorem}
  \(\lambda_{\eta}\) is a measure.
\end{theorem}
\begin{proof}
  We check (1) measure under empty set is zero and (2) countable additivity,
  which will satisfy the requirements of a measure.
  \begin{enumerate}
    \item[(1)]
      Observe that \(\lambda_{\eta} \parens{\emptyset} = 0\) because the sum
      will be trivial.

    \item[(2)]
      Let \(L_0, L_1, L_2, \ldots\) be a countable collection of pairwise
      disjoint languages.
      We decompose each of these languages into a countably indexed set,
      where \(L_{j, k}\) is the \(j\)th language's sub-language
      that only contains strings of length \(k\).
      In other words:
      \begin{align*}
        L_j = \bigcup_{k = 0}^{\infty} L_{j, k}
      \end{align*}
      Observe that by (de)-construction, for any fixed \(j\) and for all
      \(k_1 \neq k_2\), we have \(L_{j, k_1}\) and \(L_{j, k_2}\) are
      pairwise disjoint.

      However, we have a stronger condition because each \(L_j\) is
      assumed to be pairwise disjoint.
      Thus, for all \(j_1 \neq j_2\) and \(k_1 \neq k_2\),
      \(L_{j_1, k_1}\) and \(L_{j_2, k_2}\) are disjoint.
      Then:
      \begin{align*}
        \lambda_{\eta} \parens{\bigcup_{j = 0}^{\infty} L_j}
          &= \lambda_{\eta} \parens{\bigcup_{j = 0}^{\infty}
                            \bigcup_{k = 0}^{\infty} L_{j, k}} \\
          &= \sum_{j = 0}^{\infty}
              \lambda_{\eta} \parens{\bigcup_{k = 0}^{\infty} L_{j, k}} \\
          &= \sum_{j = 0}^{\infty}
              \sum_{k = 0}^{\infty}
              \frac{\abs{L_{j, k}}}{\abs{\Sigma^k}}
                \eta\parens{k} \\
          &= \sum_{j = 0}^{\infty} \lambda_{\eta} \parens{L_j}
      \end{align*}
  \end{enumerate}
  This shows that \(\lambda_{\eta} \) is indeed a measure.
\end{proof}


%%%%%%%%%%%%%%%%%%%%%%%%%%%%%%%%%%%%%%%%%%%%%%%%%%%%%%%%%%%%%%
\subsection{Measure 2: Extending the Above}
We may generalize \(\lambda_{\eta}\) as defined before slightly.
Recall the definition, where
\(L_0, L_1, L_2, \ldots\) again defines a partition of \(L\) by size:
\begin{align*}
  \lambda_{\eta}\parens{L}
    = \sum_{k = 0}^{\infty} \frac{\abs{L_k}}{\abs{\Sigma^k}} \eta\parens{k}
\end{align*}
Instead of dividing out by \(\abs{\Sigma^k}\) at each iteration of the sum,
we may take a countable series of measures
\(\nu = \braces{\nu_0, \nu_1, \nu_2, \ldots}\),
where each \(\nu_k\) has support on precisely \(\Sigma^k\).
Then, define \(\lambda_{\eta, \nu}\) as follows,
taking again \(L_0, L_1, L_2, \ldots\) the size partition of \(L\):
\begin{align*}
  \lambda_{\eta, \nu}
    = \sum_{k = 0}^{\infty} \nu_k \parens{L_k} \eta\parens{k}
\end{align*}
Often it's probably convenient to just assume each \(\nu_k \in N\)
to be the uniform distribution probability measure,
which gets us \(\lambda_{\eta}\) as defined above.

\begin{theorem}
  \(\lambda_{\eta, \nu}\) is a measure.
\end{theorem}
\begin{proof}
  We take a similar approach as before, and show
  (1) measure under empty set is zero and
  (2) countable additivity,
  which will show that \(\lambda_{\eta, \nu}\) is indeed a measure.

  \begin{enumerate}
    \item[(1)]
      Again, observe that \(\lambda_{\eta, \nu} \parens{\emptyset} = 0\)
      since the sum will be trivial.

    \item[(2)]
      Take \(L_0, L_1, L_2, \ldots\) to be a countable collection of
      pairwise disjoint languages.
      Implicitly define a countably indexed set,
      where we take each \(L_{j, k}\) as the \(j\)th language's
      sub-language with only strings of length \(k\).

      As with before, for \(j_1 \neq j_2\) and \(k_1 \neq k_2\),
      every \(L_{j_1, k_1}\) and \(L_{j_2, k_2}\) are pairwise disjoint.
      Then, doing the calculation:
      \begin{align*}
        \lambda_{\eta, \nu} \parens{\bigcup_{j = 0}^{\infty} L_j}
          &= \lambda_{\eta, \nu} \parens{\bigcup_{j = 0}^{\infty}
                            \bigcup_{k = 0}^{\infty} L_{j, k}} \\
          &= \sum_{j = 0}^{\infty}
              \lambda_{\eta, \nu} \parens{\bigcup_{k = 0}^{\infty} L_{j, k}} \\
          &= \sum_{j = 0}^{\infty}
              \sum_{k = 0}^{\infty}
                \nu_k \parens{L_k} \eta\parens{k} \\
          &= \sum_{j = 0}^{\infty} \lambda_{\eta, \nu} \parens{L_j}
        \end{align*}
      \(\lambda_{\eta, \nu}\) is therefore a measure.
  \end{enumerate}
\end{proof}

