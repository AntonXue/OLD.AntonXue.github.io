\section{Measures}
Given a family of languages \(\mcal{L}\),
let \(\sigma\parens{\mcal{L}}\) be the \(\sigma\)-algebra generated on
\(\mcal{L}\) satisfying the following:

\begin{enumerate}
  \item[(1)]
    \begin{align*}
      \emptyset, \Sigma^\star \in \sigma\parens{\mcal{L}}
    \end{align*}

  \item[(2)]
    \begin{align*}
      L \in \sigma\parens{\mcal{L}}
        \implies
          L^c = \Sigma^\star \setminus L \in \sigma\parens{\mcal{L}}
    \end{align*}

  \item[(3)]
    \begin{align*}
      L_1, L_2, \ldots \in \sigma\parens{\mcal{L}}
        \implies
          \bigcup_{k = 1}^{\infty} L_k \in \sigma\parens{\mcal{L}}
    \end{align*}

\end{enumerate}

Then \(\parens{\mcal{L}, \sigma\parens{\mcal{L}}}\) is a measurable space.

\begin{remark}
If \(\mcal{L}\) happened to be a family of regular languages,
there is no guarantee that \(\sigma\parens{\mcal{L}}\)
will still be a family of regular languages.
A counter example is the following:
\begin{align*}
  L_1 = \braces{ab}
  \qquad
  L_2 = \braces{aabb}
  \qquad
  L_3 = \braces{aaabbb}
  \qquad
  \ldots
  \qquad
  L_k = \braces{a^k b^k}
  \qquad
  \ldots
\end{align*}
But taking the countable union yields:
\begin{align*}
  \bigcup_{k = 1}^{\infty} L_k = \braces{a^n b^n \st n \in \Zz}
\end{align*}
Which is not regular.
\hfill\(\square\)
\end{remark}


\subsection{Defining Measures}

\subsubsection{From Non-negative Integers}
We first consider the non-negative integers \(\Zz\).
Let \(\eta\) be a \(\sigma\)-finite measure on \(\Zz\).
The \(\sigma\)-finite conditions ensures that no strange singularities
occur for any integers under consideration.
We may later restrict \(\eta\) to be finite if necessary,
if we want nicer conditions.

Observe that, by abuse of notation:
\begin{align*}
  \Sigma^\star =
    \bigcup_{k = 1}^{\infty} \Sigma^k
\end{align*}
In English: \(\Sigma^\star\) is the union of the set (language) of finite
strings of length \(k\), denoted \(\Sigma^k\).

Because we assumed \(\abs{\Sigma} < \infty\), this also means that:
\(\abs{\Sigma^k} = \abs{\Sigma}^k\).

Consider now some language \(L \in \sigma\parens{\mcal{L}}\).
Also decompose \(L\) into disjoint sub-languages by length as follows:
\begin{align*}
  L = \bigcup_{k = 1}^{\infty} L_k
\end{align*}
Of course, \(L_k \subseteq \Sigma^k\).

Because we are able to precisely calculate \(\abs{\Sigma^k}\),
one ``natural'' way of defining a measure \(\lambda\) on
the measurable space \(\parens{\mcal{L}, \sigma\parens{\mcal{L}}}\)
is as follows:
\begin{align*}
  \lambda\parens{L}
    = \sum_{k = 1}^{\infty} \lambda\parens{L_k}
    = \sum_{k = 1}^{\infty} \frac{\abs{L_k}}{\abs{\Sigma^k}} \eta\parens{k}
\end{align*}

