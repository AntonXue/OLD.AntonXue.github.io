\subsection{Counting}
How many unique strings of length \(k\) does an automata have?
The question is relatively straightforward for a DFA,
and slightly more complicated for a NFA.

\begin{theorem}
  There exists a polynomial-time algorithm that counts the number of strings
  accepted by a DFA.
\end{theorem}
\begin{proof}
  Consider the matrix \(M\) representation of a
  DFA \(A = \parens{\Sigma, Q, \delta, q_1, F}\),
  which we claim can be conjured in polynomial time.
  We define \(M^\prime\)
  where \(M^\prime _{i, j} = 1_{M_{i, j} \neq \emptyset}\).
  In other words, \(M^\prime\) is the adjacency
  matrix corresponding to the directed graph described by \(A\) and \(M\).

  Let \(u\) be the vector where each element is such that
  \(u_i = 1_{q_i \in F}\), then:
  \begin{align*}
    \parens{1, 0, 0, \ldots} \parens{M^\prime}^k u
  \end{align*}
  Will count the number of strings of length \(k\).
  The algorithm runs in time polynomial to \(k\) the length
  and \(n = \abs{Q}\) the number of states,
  because matrix multiplication is polynomial in complexity.
\end{proof}



