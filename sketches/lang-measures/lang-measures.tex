\documentclass[12pt]{article}

% Packages
\usepackage[margin=5em]{geometry} % 1 cm = 2.84528 em
\usepackage[backend=bibtex]{biblatex}
\bibliography{sources}
% \nocite{*}

\usepackage{lipsum}

% Paragraphs
\setlength{\parindent}{0em}
\setlength{\parskip}{1em}

% Includes
% Includes
\usepackage{amsfonts}
\usepackage{amsmath}
\usepackage{amssymb}
\usepackage{amsthm}
\usepackage{comment}
\usepackage[colorlinks]{hyperref}
\usepackage{letltxmacro}
\usepackage{listings}
\usepackage{stmaryrd}

% Spacing
\let\uspace\undefined
\newcommand{\uspace}{\ensuremath{\ }}

% Numbers
\let\N\undefined
\newcommand{\N}{\ensuremath{\mathbb{N}}}

\let\Z\undefined
\newcommand{\Z}{\ensuremath{\mathbb{Z}}}

\let\Zz\undefined
\newcommand{\Zz}{\ensuremath{\mathbb{Z}^{\geq 0}}}

\let\Zp\undefined
\newcommand{\Zp}{\ensuremath{\mathbb{Z}^{+}}}

\let\Q\undefined
\newcommand{\Q}{\ensuremath{\mathbb{Q}}}

\let\R\undefined
\newcommand{\R}{\ensuremath{\mathbb{R}}}

\let\Rz\undefined
\newcommand{\Rz}{\ensuremath{\mathbb{R}^{\geq 0}}}

\let\Rp\undefined
\newcommand{\Rp}{\ensuremath{\mathbb{R}^{+}}}

\let\C\undefined
\newcommand{\C}{\ensuremath{\mathbb{C}}}

\let\mcal\undefined
\newcommand{\mcal}[1]{\ensuremath{\mathcal{#1}}}

\let\mbb\undefined
\newcommand{\mbb}[1]{\ensuremath{\mathbb{#1}}}

% Grouping
\let\parens\undefined
\newcommand{\parens}[1]{\ensuremath{\left(#1\right)}}

\let\brackets\undefined
\newcommand{\brackets}[1]{\ensuremath{\left[#1\right]}}

\let\braces\undefined
\newcommand{\braces}[1]{\ensuremath{\left\{#1\right\}}}

\let\angles\undefined
\newcommand{\angles}[1]{\ensuremath{\left\langle#1\right\rangle}}

\let\ceil\undefined
\newcommand{\ceil}[1]{\ensuremath{\left\lceil#1\right\rceil}}

\let\floor\undefined
\newcommand{\floor}[1]{\ensuremath{\left\lfloor#1\right\rfloor}}

% Sets and Function Spaces
\let\type\undefined
\newcommand{\type}[3]{\ensuremath{#1 \colon #2 \to #3}}

\let\powset\undefined
\newcommand{\powset}[1]{\ensuremath{2^{#1}}}

\let\mod\undefined
\newcommand{\mod}{\ensuremath{\uspace\mathrm{mod}\uspace}}

\let\ker\undefined
\newcommand{\ker}{\ensuremath{\mathrm{ker}}}

\let\dom\undefined
\newcommand{\dom}{\ensuremath{\mathrm{dom}}}

\let\ran\undefined
\newcommand{\ran}{\ensuremath{\mathrm{ran}}}

\let\im\undefined
\newcommand{\im}{\ensuremath{\mathrm{im}}}

\let\coker\undefined
\newcommand{\coker}{\ensuremath{\mathrm{coker}}}

\let\codim\undefined
\newcommand{\codim}{\ensuremath{\mathrm{codim}}}

% Complex Numbers
\let\Re\undefined
\newcommand{\Re}{\ensuremath{\mathrm{Re}}}

\let\Im\undefined
\newcommand{\Im}{\ensuremath{\mathrm{Im}}}

% Probability
\let\Pr\undefined
\newcommand{\Pr}{\ensuremath{\mathrm{Pr}}}

\let\E\undefined
\newcommand{\E}{\ensuremath{\mathrm{E}}}

\let\Var\undefined
\newcommand{\Var}{\ensuremath{\mathrm{Var}}}

\let\Cov\undefined
\newcommand{\Cov}{\ensuremath{\mathrm{Cov}}}

% Integrals
\let\dee\undefined
\newcommand{\dee}[1]{\ensuremath{\uspace d #1}}

% Linear Algebra
\let\abs\undefined
\newcommand{\abs}[1]{\ensuremath{\left\lvert#1\right\rvert}}

\let\norm\undefined
\newcommand{\norm}[1]{\ensuremath{\left\lVert#1\right\rVert}}


% Computational Complexity
\let\class\undefined % Complexity class
\newcommand{\class}[1]{\ensuremath{\mathbf{#1}}}

\let\prob\undefined % Complexity problem
\newcommand{\prob}[1]{\ensuremath{\text{#1}}}

% Theorem environment
\theoremstyle{plain}
\newtheorem{theorem}{Theorem}

\theoremstyle{plain}
\newtheorem{lemma}{Lemma}

\theoremstyle{plain}
\newtheorem{claim}{Claim}

\theoremstyle{plain}
\newtheorem{fact}{Fact}

\theoremstyle{plain}
\newtheorem{remark}{Remark}

\theoremstyle{plain}
\newtheorem{definition}{Definition}

\theoremstyle{plain}
\newtheorem{example}{Example}

\theoremstyle{plain}
\newtheorem{question}{Question}

% Code environment
\lstdefinestyle{plainsty}{
  basicstyle=\small\ttfamily,
  language=C,
  xleftmargin=\parindent,
  aboveskip=1em,
  belowskip=1em,
  showspaces=false,
  showstringspaces=false,
  keywordstyle = {},
}

\lstnewenvironment{pcode}{\lstset{style=plainsty}}{}

\let\pinl\undefined
\newcommand{\pinl}{\lstinline[style=plainsty]}

\newcommand*{\SavedLstInline}{} % Allows plain code usage in math mode.
\LetLtxMacro\SavedLstInline\pinl
\DeclareRobustCommand*{\pinl}{%
  \ifmmode
    \let\SavedBGroup\bgroup
    \def\bgroup{%
      \let\bgroup\SavedBGroup
      \hbox\bgroup
    }%
  \fi
\SavedLstInline}

\let\ttcode\undefined
\newcommand{\ttcode}[1]{\small{\texttt{#1}}}


% Text markings

\let\tturl\undefined
\newcommand{\tturl}[1]{\href{#1}{\texttt{#1}}}

\let\red\undefined
\newcommand{\red}[1]{\textbf{\color{red}#1}}


% Abbreviations
\let\st\undefined
\newcommand{\st}{\ensuremath{\uspace\colon\uspace}}

\let\ow\undefined
\newcommand{\ow}{\ensuremath{\text{otherwise}}}


\usepackage{stmaryrd}


\let\usp\undefined
\newcommand{\usp}{\>}

% Grouping
\let\parens\undefined
\newcommand{\parens}[1]{\text{(}#1\text{)}}

\let\brackets\undefined
\newcommand{\brackets}[1]{\text{[}#1\text{]}}

\let\braces\undefined
\newcommand{\braces}[1]{\text{\{}#1\text{\}}}


% Language

\let\regex\undefined
\newcommand{\regex}{\ensuremath{E}}

\let\regexes\undefined
\newcommand{\regexes}{\ensuremath{\mathcal{E}}}

\let\reglang\undefined
\newcommand{\reglang}{\ensuremath{L_{R}}}

\let\reglangs\undefined
\newcommand{\reglangs}{\ensuremath{\mathcal{L}_R}}

\let\lang\undefined
\newcommand{\lang}{\ensuremath{L}}

\let\langs\undefined
\newcommand{\langs}{\ensuremath{\mathcal{L}}}

\let\alphabet\undefined
\newcommand{\alphabet}{\ensuremath{\Sigma}}

\let\empstr\undefined
\newcommand{\empstr}{\ensuremath{\varepsilon}}

\let\langmin\undefined
\newcommand{\langmin}{\ensuremath{\eta}}


% Measures

\let\mucount\undefined
\newcommand{\mucount}{\ensuremath{\mu_{c}}}

\let\premucount\undefined
\newcommand{\premucount}{\ensuremath{\mu_{0, c}}}






% Author
\title{Measures on Languages}
% \author{Anton Xue}
% \date{\today}
\date{}

% Document
\begin{document}
\maketitle

\section{Introduction}

\subsection{Notation}

Let \(\Sigma\) denote a non-empty, countable alphabet.
Unless otherwise specified, assume \(\abs{\Sigma} < \infty\).
Write \(\epsilon\) to mean the empty string.

Let \(\Sigma^\star\) be the set of all finite strings from \(\Sigma\).
Write strings as \(w\) or \(s\), whichever happens to be more convenient.

Let \(L\) denote a language, implicitly over \(\Sigma\).
In other words, \(L \subseteq \Sigma^\star\).

We treat the empty language \(\emptyset\) as distinct from
the language with a single empty string \(\braces{\epsilon}\).

Let \(\mcal{L}\) be a family of languages.

Write deterministic finite automatons shorthand as DFA,
and non-deterministic finite automatons shorthand as NFA.

Write regular expressions shorthand as regex.

If \(X\) is a set, then \(\powset{X}\) is the powerset of \(X\).

Unless otherwise noted, vectors are implicitly in column format.


%%%%%%%%%%%%%%%%%%%%%%%%%%%%%%%%%%%%%%%%%%%%
\subsection{Regular Expressions}
A regular expression over an alphabet \(\Sigma\) describes regular
languages over \(\Sigma\).
Regular expressions are inductively generated,
and we borrow heavily from Savage~\cite{savage1998models}.

\begin{definition}[Regular Expression]
  A regular expression over the finite alphabet \(\Sigma\)
  is defined inductively:
  \begin{enumerate}
    \item[(1)]
      The empty language \(\emptyset\) is a regular expression.

    \item[(2)]
      The empty string \(\epsilon\) is a regular expression denoting
      \(\braces{\epsilon}\).

    \item[(3)]
      For each \(a \in \Sigma\),
      the standalone \(a\) is a regular expression
      denoting the singleton set \(\braces{a}\).

    \item[(4)]
      If \(r\) and \(s\) are regular expressions, then so are
      \(rs\) (string concat), \(r + s\) (string choice),
      and \(r^\star\) (string repeat).
  \end{enumerate}
\end{definition}

\begin{theorem}[Regular Expression Axioms]
Regular expressions satisfy the following axioms:
\begin{enumerate}
  \item[(1)]
    \(r \emptyset = \emptyset r = \emptyset\)

  \item[(2)]
    \(r \epsilon = \epsilon r = r\)

  \item[(3)]
    \(r + \emptyset = \emptyset + r = r\)

  \item[(4)]
    \(r + r = r\)

  \item[(5)]
    \(r + s = s + r\)

  \item[(6)]
    \(r\parens{s + t} = rs + rt\)

  \item[(7)]
    \(\parens{r + s}t = rt + st\)

  \item[(8)]
    \(r(st) = (rs)t\)

  \item[(9)]
    \(\emptyset^\star = \epsilon\)

  \item[(10)]
    \(\epsilon^\star = \epsilon\)

  \item[(11)]
    \(\parens{\epsilon + r}^+ = r^\star\)

  \item[(12)]
    \(\parens{\epsilon + r}^\star = r^\star\)

  \item[(13)]
    \(r^\star \parens{\epsilon + r} = \parens{\epsilon + r}r^\star = r^\star\)

  \item[(14)]
    \(r^\star s + s = r^\star s\)

  \item[(15)]
    \(r\parens{sr}^\star = \parens{rs}^\star r\)

  \item[(16)]
    \(\parens{r + s}^\star = \parens{r^\star s}^\star r^\star = \parens{s^\star r}^\star s^\star\)

\end{enumerate}

\end{theorem}

Outside of the Kleene star \(\star\) operation, axioms (1 - 8)
effectively state that regular expressions are an
idempotent semiring with additive constant \(\emptyset\) and multiplicative
constant \(\epsilon\)~\cite{infoandcomp-kozen1994kleenealg}.



%%%%%%%%%%%%%%%%%%%%%%%%%%%%%%%%%%%%%%%%%%%%
\subsection{Representation of Regular Languages}
There are several ways to represent regular languages,
many of which can be found in literature~\cite{savage1998models}.
We work with whatever is convenient for the problem at hand.
For a regular language \(L\) over an alphabet \(\Sigma\),
there are a few notable ones:


\begin{enumerate}
  \item[(1)]
    \textbf{Sets}:
    sometimes if the language is finite or has a simple structure,
    a complete set presentation may be convenient.

  \item[(2)]
    \textbf{Regular expressions}:
    compact representation,
    also commonly used in practice when trying to do string matching.

  \item[(3)]
    \textbf{Finite state machines}:
    Savage~\cite{savage1998models} gives a fairly standard representation.

    \begin{definition}[DFA]
      A DFA \(A\) is a five-tuple \(A = \parens{\Sigma, Q, \delta, q_0, F}\),
      where \(\Sigma\) is the alphabet, \(Q\) is the finite set of states,
      \(\type{\delta}{Q \times \Sigma}{Q}\) is the transition function,
      \(q_0\) is the initial state, and \(F\) is the set of final states.
    \end{definition}

    For convenience, we might also write \(q_0\) as \(q_1\),
    especially when talking about matrix indices.
    We'll try to remember to make note of when this rewriting is done.

    \begin{definition}[NFA]
      A NFA \(A\) is identically defined except for the
      transition function, which is now
      \(\type{\delta}{Q \times \Sigma}{\powset{Q}}\).
      Each transition non-deterministically picks one state from the set.
    \end{definition}

  \item[(4)]
    \textbf{Matrices}:
    transition matrices can be constructed from both DFAs and NFAs.
    First, take \(Q = \braces{q_1, q_2, \ldots, q_n}\).
    There are two primary possibilities:

    \begin{enumerate}
      \item[(a)]
        A matrix \(M_A\) corresponding to an automata \(A\),
        with \(q_1\) the initial state.
        Write \(+\braces{\ldots}\) do denote
        the summation regular expression over a set.
        We construct the matrix as follows:
        \begin{align*}
          M_{A, i, j} =
            + \braces{a \st \parens{\parens{q_i, a}, q_j} \in \delta}
        \end{align*}
        Here \(a\) is any character of \(\Sigma\).
        In short, each entry of the matrix \(M_{A, i, j}\)
        is the \(+\) of all the characters
        that permit the transition from \(q_i\) to \(q_j\).

        If we take \(v = (\parens{1, 0, \ldots}\) as an \(n\)-dimensional
        vector, where each coordinate \(i\) represents state \(q_i\).
        Suppose that \(u\) is an \(n\)-dimensional vector indicating
        the final states,
        where \(u_i = 1_{q_i \in F}\), then:
        \begin{align*}
          \transp{v} M_A ^k u
        \end{align*}
        Will corresponds to the regular expression of the sub-language
        of strings of precisely length \(k\).

      \item[(b)]
        Alternatively we may see regular expressions as a
        set of matrices, each corresponding to a letter of \(\Sigma\).
        In essence, for each \(a \in \Sigma\), the associated
        matrix \(M_a\) has form:
        \(M_{a, i, j} = 1_{\parens{\parens{q_i, \cdot}, q_j} \in \delta}\),
        and indicates an adjacency transition matrix.

        The matrix described earlier can be recovered by observing that:
        \begin{align*}
          M_{A} = \sum_{a \in \Sigma} a M_{a}
        \end{align*}

    \end{enumerate}

\end{enumerate}

%%%%%%%%%%%%%%%%%%%%%%%%%%%%%%%%%%%%%%%%%%%
\section{Measures on Languages}
Given a family of languages \(\mcal{L}\),
let \(\sigma\parens{\mcal{L}}\) be the \(\sigma\)-algebra generated on
\(\mcal{L}\) satisfying the following:

\begin{enumerate}
  \item[(1)]
    \begin{align*}
      \emptyset, \Sigma^\star \in \sigma\parens{\mcal{L}}
    \end{align*}

  \item[(2)]
    \begin{align*}
      L \in \sigma\parens{\mcal{L}}
        \implies
          L^c = \Sigma^\star \setminus L \in \sigma\parens{\mcal{L}}
    \end{align*}

  \item[(3)]
    \begin{align*}
      L_0, L_1, \ldots \in \sigma\parens{\mcal{L}}
        \implies
          \bigcup_{k = 0}^{\infty} L_k \in \sigma\parens{\mcal{L}}
    \end{align*}

\end{enumerate}

Then \(\parens{\mcal{L}, \sigma\parens{\mcal{L}}}\) is a measurable space.

\begin{remark}
If \(\mcal{L}\) happened to be a family of regular languages,
there is no guarantee that \(\sigma\parens{\mcal{L}}\)
will still be a family of regular languages.
A counter example is the following:
\begin{align*}
  L_0 = \braces{\varepsilon}
  \qquad
  L_1 = \braces{ab}
  \qquad
  L_2 = \braces{aabb}
  \qquad
  \ldots
  \qquad
  L_k = \braces{a^k b^k}
  \qquad
  \ldots
\end{align*}
But taking the countable union yields:
\begin{align*}
  \bigcup_{k = 0}^{\infty} L_k = \braces{a^k b^k \st k \in \Zz}
\end{align*}
Which is not regular.
\end{remark}


%%%%%%%%%%%%%%%%%%%%%%%%%%%%%%%%%%%%%%%%%%%%%%%%%%%%%%%%%%%%%%
\subsection{Measure 1: From Non-negative Integers}
We first consider the non-negative integers \(\Zz\).
Let \(\eta\) be a \(\sigma\)-finite measure on \(\Zz\).
The \(\sigma\)-finite conditions ensures that no strange singularities
occur for any integers under consideration.
We may later restrict \(\eta\) to be finite if necessary,
if we want nicer conditions.

Observe that, by abuse of notation:
\begin{align*}
  \Sigma^\star =
    \bigcup_{k = 0}^{\infty} \Sigma^k
\end{align*}
In English: \(\Sigma^\star\) is the union of the set (language) of finite
strings of length \(k\), denoted \(\Sigma^k\).

Because we assumed \(\abs{\Sigma} < \infty\), this also means that:
\(\abs{\Sigma^k} = \abs{\Sigma}^k\).

Consider now some language \(L \in \sigma\parens{\mcal{L}}\).
Also decompose \(L\) into disjoint sub-languages by length as follows,
with convenient subscripting:
\begin{align*}
  L = \bigcup_{k = 0}^{\infty} L_k
\end{align*}
Of course, \(L_k \subseteq \Sigma^k\).

Because we are able to precisely calculate \(\abs{\Sigma^k}\),
one ``natural'' way of defining a measure \(\lambda_{\eta}\) on
the measurable space \(\parens{\mcal{L}, \sigma\parens{\mcal{L}}}\)
is as follows:
\begin{align*}
  \lambda_{\eta} \parens{L}
    = \sum_{k = 0}^{\infty} \lambda_{\eta} \parens{L_k}
    = \sum_{k = 0}^{\infty} \frac{\abs{L_k}}{\abs{\Sigma^k}} \eta\parens{k}
\end{align*}

We claim that
\(\parens{\mcal{L}, \sigma\parens{\mcal{L}}, \lambda_{\eta}}\)
forms a measure space.

\begin{theorem}
  \(\lambda_{\eta}\) is a measure.
\end{theorem}
\begin{proof}
  We check (1) measure under empty set is zero and (2) countable additivity,
  which will satisfy the requirements of a measure.
  \begin{enumerate}
    \item[(1)]
      Observe that \(\lambda_{\eta} \parens{\emptyset} = 0\) because the sum
      will be trivial.

    \item[(2)]
      Let \(L_0, L_1, L_2, \ldots\) be a countable collection of pairwise
      disjoint languages.
      We decompose each of these languages into a countably indexed set,
      where \(L_{j, k}\) is the \(j\)th language's sub-language
      that only contains strings of length \(k\).
      In other words:
      \begin{align*}
        L_j = \bigcup_{k = 0}^{\infty} L_{j, k}
      \end{align*}
      Observe that by (de)-construction, for any fixed \(j\) and for all
      \(k_1 \neq k_2\), we have \(L_{j, k_1}\) and \(L_{j, k_2}\) are
      pairwise disjoint.

      However, we have a stronger condition because each \(L_j\) is
      assumed to be pairwise disjoint.
      Thus, for all \(j_1 \neq j_2\) and \(k_1 \neq k_2\),
      \(L_{j_1, k_1}\) and \(L_{j_2, k_2}\) are disjoint.
      Then:
      \begin{align*}
        \lambda_{\eta} \parens{\bigcup_{j = 0}^{\infty} L_j}
          &= \lambda_{\eta} \parens{\bigcup_{j = 0}^{\infty}
                            \bigcup_{k = 0}^{\infty} L_{j, k}} \\
          &= \sum_{j = 0}^{\infty}
              \lambda_{\eta} \parens{\bigcup_{k = 0}^{\infty} L_{j, k}} \\
          &= \sum_{j = 0}^{\infty}
              \sum_{k = 0}^{\infty}
              \frac{\abs{L_{j, k}}}{\abs{\Sigma^k}}
                \eta\parens{k} \\
          &= \sum_{j = 0}^{\infty} \lambda_{\eta} \parens{L_j}
      \end{align*}
  \end{enumerate}
  This shows that \(\lambda_{\eta} \) is indeed a measure.
\end{proof}


%%%%%%%%%%%%%%%%%%%%%%%%%%%%%%%%%%%%%%%%%%%%%%%%%%%%%%%%%%%%%%
\subsection{Measure 2: Extending the Above}
We may generalize \(\lambda_{\eta}\) as defined before slightly.
Recall the definition, where
\(L_0, L_1, L_2, \ldots\) again defines a partition of \(L\) by size:
\begin{align*}
  \lambda_{\eta}\parens{L}
    = \sum_{k = 0}^{\infty} \frac{\abs{L_k}}{\abs{\Sigma^k}} \eta\parens{k}
\end{align*}
Instead of dividing out by \(\abs{\Sigma^k}\) at each iteration of the sum,
we may take a countable series of measures
\(\nu = \braces{\nu_0, \nu_1, \nu_2, \ldots}\),
where each \(\nu_k\) has support on precisely \(\Sigma^k\).
Then, define \(\lambda_{\eta, \nu}\) as follows,
taking again \(L_0, L_1, L_2, \ldots\) the size partition of \(L\):
\begin{align*}
  \lambda_{\eta, \nu}
    = \sum_{k = 0}^{\infty} \nu_k \parens{L_k} \eta\parens{k}
\end{align*}
Often it's probably convenient to just assume each \(\nu_k \in N\)
to be the uniform distribution probability measure,
which gets us \(\lambda_{\eta}\) as defined above.

\begin{theorem}
  \(\lambda_{\eta, \nu}\) is a measure.
\end{theorem}
\begin{proof}
  We take a similar approach as before, and show
  (1) measure under empty set is zero and
  (2) countable additivity,
  which will show that \(\lambda_{\eta, \nu}\) is indeed a measure.

  \begin{enumerate}
    \item[(1)]
      Again, observe that \(\lambda_{\eta, \nu} \parens{\emptyset} = 0\)
      since the sum will be trivial.

    \item[(2)]
      Take \(L_0, L_1, L_2, \ldots\) to be a countable collection of
      pairwise disjoint languages.
      Implicitly define a countably indexed set,
      where we take each \(L_{j, k}\) as the \(j\)th language's
      sub-language with only strings of length \(k\).

      As with before, for \(j_1 \neq j_2\) and \(k_1 \neq k_2\),
      every \(L_{j_1, k_1}\) and \(L_{j_2, k_2}\) are pairwise disjoint.
      Then, doing the calculation:
      \begin{align*}
        \lambda_{\eta, \nu} \parens{\bigcup_{j = 0}^{\infty} L_j}
          &= \lambda_{\eta, \nu} \parens{\bigcup_{j = 0}^{\infty}
                            \bigcup_{k = 0}^{\infty} L_{j, k}} \\
          &= \sum_{j = 0}^{\infty}
              \lambda_{\eta, \nu} \parens{\bigcup_{k = 0}^{\infty} L_{j, k}} \\
          &= \sum_{j = 0}^{\infty}
              \sum_{k = 0}^{\infty}
                \nu_k \parens{L_k} \eta\parens{k} \\
          &= \sum_{j = 0}^{\infty} \lambda_{\eta, \nu} \parens{L_j}
        \end{align*}
      \(\lambda_{\eta, \nu}\) is therefore a measure.
  \end{enumerate}
\end{proof}


\section{Approximating Languages}
Given a measure space
\(\parens{\mcal{L}, \sigma\parens{L}, \lambda}\),
we may consider how similar two languages are.
For two languages \(L_1, L_2 \in \sigma\parens{\mcal{L}}\),
a ``natural'' difference is to consider their
symmetric set difference:
\begin{align*}
  d\parens{L_1, L_2}
    = \lambda\parens{L_1 \triangle L_2}
    = \lambda\parens{\parens{L_1 \setminus L_2} \cup \parens{L_2 \setminus L_1}}
\end{align*}
Recall that by the definition of a \(\sigma\)-algebra,
the symmetric set difference \(L_1 \triangle L_2\)
is in \(\sigma\parens{\mcal{L}}\), and therefore measurable.

We hope restrict our attention to regular languages for now,
or in other words, the class of languages precisely recognized by
DFAs, and ask the following:

\begin{question}
  Given a regular language \(L\) recognized by a minimal DFA \(A\)
  and some \(\varepsilon > 0\),
  does there exist a regular language \(L^\prime\) recognized by \(A^\prime\)
  such that \(A^\prime\) has less states than \(A\),
  and \(d\parens{L, L^\prime} < \varepsilon\)?
\end{question}


\begin{example}
  Consider \(\Sigma = \braces{a}\),
  where \(L_1 = aa^\star\) and \(L_2 = aaa^\star\),
  and assume a geometric probability measure for \(\eta\)
  with success of probability \(0 < p \leq 1\),
  through which \(\lambda\) is defined.

  Recall that for the geometric distribution where \(n \in \Zp\):
  \begin{align*}
    \eta\parens{n} = \parens{1 - p}^{n - 1} p
  \end{align*}
  Observe that for \(L_1\) and \(L_2\), we have:
  \begin{align*}
    L_1 &=
      \underbrace{\braces{}}_{\text{length \(0\)}} \cup
      \underbrace{\braces{a}}_{\text{length \(1\)}} \cup
      \underbrace{\braces{aa}}_{\text{length \(2\)}} \cup
      \underbrace{\braces{aaa}}_{\text{length \(3\)}} \cup
      \ldots \\
    L_2 &= 
      \underbrace{\braces{}}_{\text{length \(0\)}} \cup
      \underbrace{\braces{}}_{\text{length \(1\)}} \cup
      \underbrace{\braces{aa}}_{\text{length \(2\)}} \cup
      \underbrace{\braces{aaa}}_{\text{length \(3\)}} \cup
      \ldots
  \end{align*}
  In other words, the only set on which the two languages differ is
  strings of length \(1\), which \(L_1\) has, but \(L_2\) does not.
  For the difference, this then means that:
  \begin{align*}
    d\parens{L_1, L_2}
      = \lambda\parens{L_1 \triangle L_2}
      = \lambda\parens{\braces{a}}
      = \frac{\abs{\braces{a}}}{\abs{\Sigma^k}} \eta \parens{1}
      = \frac{1}{1} p = p
  \end{align*}
  In other words, with probability \(p\),
  we can distinguish random strings generated from \(L_1\) and \(L_2\),
  where the probability distribution is geometric,
  and over the length of the strings.
  Selection of the strings once length is fixed is irrelevant because
  each set corresponding to a length contains only one string.
\end{example}

Assume a (regular) language \(L\) given as an automata or regular
expression, whichever may be convenient, and measure \(\lambda_{n}\).
A few challenges lie ahead:

\begin{question}
  How can we quickly measure \(\lambda_{\eta}\parens{L}\)?
\end{question}

\begin{question}
  How can we quickly generate a word of some length from \(L\)?
\end{question}

The word ``quickly'' here is key:
many methods that can be thrown together to answer these questions
are intrinsically exponential,
so if we can introduce sub-exponential time techniques
that would be pretty cool.


%%%%%%%%%%%%%%%%%%%%%%55
\subsection{Counting}
How many unique strings of length \(k\) does an automata have?
The question is relatively straightforward for a DFA,
and slightly more complicated for a NFA.

\begin{theorem}
  There exists a polynomial-time algorithm that counts the number of strings
  accepted by a DFA.
\end{theorem}
\begin{proof}
  Consider the matrix \(M\) representation of a
  DFA \(A = \parens{\Sigma, Q, \delta, q_1, F}\),
  which we claim can be conjured in polynomial time.
  Roughly, the sketch is that we can enumerate each element
  of the transition \(\delta\) and iteratively populate our matrix \(M\).

  We define \(M^\prime\)
  where \(M^\prime _{i, j} = 1_{M_{i, j} \neq \emptyset}\).
  In other words, \(M^\prime\) is the adjacency
  matrix corresponding to the directed graph described by \(A\) and \(M\).

  Let \(u\) be the vector where each element is such that
  \(u_i = 1_{q_i \in F}\), then:
  \begin{align*}
    \parens{1, 0, 0, \ldots} \parens{M^\prime}^k u
  \end{align*}
  Will count the number of strings of length \(k\).
  The algorithm runs in time polynomial to \(k\) the length
  and \(n = \abs{Q}\) the number of states,
  because matrix multiplication is polynomial in complexity.
\end{proof}

However, counting the number of strings of length \(k\) is known to be
\(\class{\#P}\)~\cite{acmsiam-kannan1995counting},
and exponential-time algorithms are known:
just reduce the NFA to a DFA and run the solution above.
Indeed, this is not very satisfying, but at least it's something.

\subsection{Approximate Counting}
If counting is hard, then perhaps approximate counting is easier?
Indeed, this may be the step towards fast comparison of regular languages.
To do this,
we attempt to adapt work by Kannan~\cite{acmsiam-kannan1995counting},
which we have cited quite a few times now.

Kannan's construction presents a few ``proofs left to the reader'',
which we complete here.
However, we first present a few definitions that are used throughout
the paper.

\begin{definition}[\(g\parens{n}\)-randomized approximation scheme]
  A \(g\parens{n}\)-ras for a non-negative real-valued function
  \(f\) is a probabilistic algorithm which,
  on input \(x\) and \(\varepsilon > 0\) and \(\delta < 1\),
  computes \(\widetilde{f}\parens{x}\) where:
  \begin{align*}
    1 - \delta \leq
    \Pr\brackets{
      f\parens{x} \parens{1 + \varepsilon}^{-1}
        \leq \widetilde{f}\parens{x}
        \leq f\parens{x} \parens{1 + \varepsilon}}
  \end{align*}
  Further, the algorithm runs in expected time \(O\parens{g\parens{n}}\)
  where:
  \begin{align*}
    n = \max\braces{\abs{x}, \varepsilon^{-1}, \log\parens{\delta^-1}}
  \end{align*}
\end{definition}

In other words, this randomized approximation scheme is pretty tight.

\begin{definition}[\(g\parens{n}\)-almost uniform generator]
  Let \(R\) be a polynomial-time computable binary relation.
  For any \(x\), let
  \begin{align*}
    \phi\parens{x} = \braces{y \st \parens{x, y} \in R}
  \end{align*}
  A \(g\parens{n}\)-almost uniform generator for the relation
  \(R\) is a probabilistic algorithm \(A\),
  which, on input \(x\), \(\varepsilon > 0\),
  outputs some \(y \in \phi\parens{x}\) such that:
  \begin{align*}
    \abs{\phi\parens{x}}^{-1} \parens{1 + \varepsilon}^{-1}
      \leq \Pr\brackets{A\parens{x, \varepsilon} = y}
      \leq \abs{\phi\parens{x}}^{-1} \parens{1 + \varepsilon}
  \end{align*}
  Also, \(A\) runs in time \(O\parens{g\parens{n}}\)
  where:
  \begin{align*}
    n = \max\braces{\abs{x}, \log\parens{\varepsilon^{-1}}}
  \end{align*}
\end{definition}

\begin{definition}[\(m\)-level NFA]
  An \(m\)-level NFA is an NFA in which the states can be
  partitioned into \(m + 1\) levels with the following properties:
  \begin{enumerate}
    \item[(1)]
      There is exactly one state at level \(0\) and it is the start state.

    \item[(2)]
      There is exactly one state at level \(m\) and it is the accept state.

    \item[(3)]
      All transitions are from a node in level \(i\) to a node in
      level \(i + 1\) for \(i \in \braces{0, \ldots, m - 1}\).

    \item[(4)]
      For any state, \(p\),
      the accept state is reachable from \(p\),
      and \(p\) is reachable from the start state
      (fully connected automata).
  \end{enumerate}
\end{definition}

Kannan further poses the following lemma as an exercise to the reader,
which we also prove:

\begin{lemma}
  For any NFA, \(M\), and integer \(m\),
  there exists an NFA \(M^\prime\) such that \(M^\prime\)
  is an \(m\)-level NFA with no \(\epsilon\)-transitions,
  a binary alphabet, and such that
  \(\abs{L\parens{M^\prime}} = \abs{L\parens{M} \cap \Sigma^m}\).
  Further, the size of \(M^\prime\) is polynomial in size of \(M\) and \(m\).
\end{lemma}
\begin{proof}
  Given an NFA \(M_1\), it is well-known that there exist
  a polynomial-time and polynomial-space reduction to an equivalent
  NFA \(M_2\) without \(\epsilon\)-transitions.

  As \(M_2\) is presumed to be defined over some alphabet \(\Sigma\),
  we now need to reduce this to an equivalent NFA over a binary alphabet
  \(\Sigma_2\).
  To do this, recall from standard results in complexity that we need
  \(\ceil{\log\parens{\abs{\Sigma}}}\) bits to represent every element
  of \(\Sigma\).
  Hence, in order to make an \(\epsilon\)-transition-free and
  binary alphabet NFA,
  we may augment every transition edge of \(M_2\)
  by a sequence of transitions of length
  \(\ceil{\log\parens{\abs{\Sigma}}}\) that now uses the binary
  alphabet \(\Sigma_2\) instead of \(\Sigma\).
  Take \(m^\prime = m \ceil{\log\parens{\abs{\Sigma}}}\).
  This produced \(M_3\), which lacks \(\varepsilon\)-transitions
  and is over a binary alphabet \(\Sigma_2\).

  To convert the NFA \(M_3\) into a
  \(m^\prime\)-level NFA,
  we do the following.
  First, let \(Q\) be the set of states of \(M_3\).
  We create \(m^\prime\) copies of \(Q\), and write them as
  \(Q_0, Q_1, \ldots, Q_{m^\prime - 1}\).
  These serve to eventually form the first \(m^\prime\)-layers out of the
  \(m^\prime + 1\) for an \(m^\prime\)-level NFA.
  For each set of states \(Q_i\), with
  \(i \in \braces{0, \ldots, m^\prime - 2}\)
  as follows: if there exists a one-step transition from some \(q_a\) to
  \(q_b\) on character \(a \in \Sigma_2\) within \(M_3\),
  then we connect \(q_a \in Q_i\) and \(q_b \in Q_{i + 1}\) using
  letter \(a\).
  We add a final layer consisting of just one state \(q_F\),
  and connect all the final states of \(Q_{m^\prime - 1}\) to
  \(q_F\) using the technique just described.
  This creates \(M_4\), which has \(m^\prime + 1\) layers,
  the last of which has a single final state.

  In order to make \(M_4\) into an \(m^\prime\)-level NFA
  (which has \(m^\prime + 1\) layers),
  we need another step of transformation.
  Take \(q_0 \in Q\) to be the start state of \(M_3\).
  Then, in the first layer, mark \(q_0\) as the start state.
  Now take the sub-graph of \(M_4\) that is reachable from \(q_0\)
  in the first layer,
  and this is the desired \(m^\prime\)-level automaton \(M_5\).

  Because the start state \(q_0\) is unique, it is clear that
  the first layer only has a single state.
  Furthermore, the last layer also has a single state \(q_F\), which
  we added by construction earlier.
  In addition, all transition edges by construction flow from
  layer \(i\) to the next layer \(i + 1\).

  Because we have taken the parts reachable from the initial state \(q_0\),
  we guarantee reachability.
  It may be the case that the final state \(q_F\) is not reachable from
  \(q_0\), in which case this is fine,
  and the \(m^\prime\)-level NFA cannot be constructed anyways,
  because it implies that no strings of length \(m\) exists in
  the language.
  Further note that \(m^\prime\) is polynomial in complexity to \(m\).
  Hence, \(M_5\) is the desired \(m^\prime\)-level NFA.

\end{proof}


%%%%%%%%%%%%%%%%%%%%%%%%%%%%%%%%%%%%%
\subsection{An Algorithm For Measuring}

We are now ready to present a
quasi-polynomial algorithm for calculating the
measure of a language's strings up to some length \(N\),
which we describe as follows:

\begin{enumerate}
  \item[(1)]
    Take as input a NFA \(M\) that represents language \(L\),
    some positive integer \(N\), and error bounds \(\varepsilon > 0\)
    and confidence \(\delta < 1\).

  \item[(2)]

    Let \(A\) be the algorithm specified in
    Kannan's work~\cite{acmsiam-kannan1995counting}
    that takes as input a \(m_A\)-level NFA,
    error tolerance \(\varepsilon_A\),
    and confidence \(\delta_A\).

  \item[(3)]
    At each stage, let \(M_m\) be the \(m\)-level NFA of \(M\), and
    calculate the following:
    \begin{align*}
      v =
      \sum_{k = 0}^{N}
        \frac{A\parens{M_k, \varepsilon / N, \delta / N}}{\abs{\Sigma^k}}
          \eta \parens{k}
    \end{align*}

  \item[(4)]
    Return \(v\) as an estimate for the measure up to
    strings of length \(N\).
\end{enumerate}


\begin{theorem}
  The algorithm runs in quasi-polynomial time with respect to
  \(N\), \(\abs{M}\), \(\varepsilon\), and \(\delta\).
\end{theorem}
\begin{proof}
  It suffices to show that each iteration of the sum is
  quasi-polynomial in complexity.
  Because only \(N\) iterations occur,
  if each iteration is quasi-polynomial,
  then the entire sum is also quasi-polynomial.

  We have shown before that a polynomial-time reduction from
  \(M\) to \(M_k\) is possible for each \(k\).
  Furthermore, this is directly fed into the
  quasi-polynomial time algorithm \(A\).
  We implicitly assume \(\eta\) to be polynomial time in complexity.
  Therefore, each iteration is quasi-polynomial time in complexity.

  Together, this means that the entire algorithm is
  quasi-polynomial time in complexity.
\end{proof}


\subsection{Metric Spaces}

\begin{lemma}
  \(\parens{A \triangle C}
          \subseteq \parens{A \triangle B} \cup \parens{B \triangle C}\).
\end{lemma}
\begin{proof}
  Observe that we may rewrite the above as follows:
  \begin{align*}
    \parens{A \setminus C} \cup \parens{C \setminus A}
      \subseteq
        \brackets{\parens{A \setminus B} \cup \parens{B \setminus C}} \cup
        \brackets{\parens{B \setminus A} \cup \parens{C \setminus B}}
  \end{align*}
  It then suffices to show that:
  \begin{align*}
    A \setminus C
      \subseteq \parens{A \setminus B} \cup \parens{B \setminus C}
    \qquad
    \qquad
    C \setminus A
      \subseteq \parens{B \setminus A} \cup \parens{C \setminus B}
  \end{align*}
  We take turns examining these
  \begin{enumerate}
    \item[(i)]
      If \(x \in A \setminus C\), then this implies that \(x \in A\)
      and \(x \not\in C\).
      There are now two cases, where \(x \in B\) or \(x \not\in B\).
      First assume that \(x \in B\), which will imply that
      \(x \in B \setminus C\).
      Now assume that \(x \not\in B\), which will imply that
      \(x \in A \setminus B\).
      Either way, the implication is that
      \(x \in \parens{A \setminus B} \cup \parens{B \setminus C}\),
      and so it follows that
      \(A \setminus C
      \subseteq \parens{A \setminus B} \cup \parens{B \setminus C}\).

    \item[(ii)]
      If \(x \in C \setminus A\), then this implies that
      \(x \in C\) and \(x \not\in A\).
      The argument is similar to the above, in which either
      \(x \in B\) or \(x \not\in B\).
      If \(x \in B\), then \(x \in B \setminus A\),
      and otherwise if \(x \not\in B\) implies that
      \(x \in C \setminus B\).
      Collectively, the two imply that
      \(C \setminus A \subseteq
        \parens{B \setminus A} \cup \parens{C \setminus B}\).
  \end{enumerate}
  Collectively, this shows that
  \(\parens{A \triangle C}
          \subseteq \parens{A \triangle B} \cup \parens{B \triangle C}\).
\end{proof}


\begin{theorem}
  If \(\parens{X, \Sigma, \lambda}\) is a measure space,
  then for \(\type{d}{\Sigma \times \Sigma}{\Rz}\) defined as:
  \begin{align*}
    d\parens{A, B} = \lambda\parens{A \triangle B}
  \end{align*}
  Is a metric function.
\end{theorem}
\begin{proof}
  We prove the conditions necessary for a metric:
  identity, symmetry, and triangle inequality.
  \begin{enumerate}
    \item[(a)]
      As \(\lambda\) is a measure, then for any \(A \in \Sigma\):
      \begin{align*}
        d\parens{A, A}
          = \lambda\parens{A \triangle A}
          = \lambda\parens{\emptyset} = 0
      \end{align*}

    \item[(b)]
      By the symmetry of symmetric set difference, for any \(A, B \in \Sigma\):
      \begin{align*}
        d\parens{A, B}
          = \lambda\parens{A \triangle B}
          = \lambda\parens{B \triangle A}
          = d\parens{B, A}
      \end{align*}

    \item[(c)]
      For any \(A, B, C \in \Sigma\),
      we have by convexity as shown in the lemma above:
      \begin{align*}
        A \triangle C \subseteq
          \parens{A \triangle B} \cup \parens{B \triangle C}
      \end{align*}
      Then by sub-additivity of measures:
      \begin{align*}
        d\parens{A, C}
          = \lambda\parens{A \triangle C}
          \leq \lambda\parens{
                \parens{A \triangle B} \cup \parens{B \triangle C}}
          \leq
          \lambda\parens{A \triangle B} + \lambda\parens{A \triangle C}
          = d\parens{A, B} + d\parens{B, C}
      \end{align*}
  \end{enumerate}
\end{proof}

\pagebreak

\printbibliography

\end{document}

