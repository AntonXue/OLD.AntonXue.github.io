\documentclass[12pt]{article}

% Packages
\usepackage[margin=5em]{geometry} % 1 cm = 2.84528 em
\usepackage[backend=bibtex]{biblatex}
\usepackage{xcolor}
\bibliography{sources}
% \nocite{*}

\usepackage{lipsum}

% Paragraphs
\setlength{\parindent}{0em}
\setlength{\parskip}{1em}

% Includes
\input{antonxue-lib.tex}
% \usepackage{stmaryrd}


\let\usp\undefined
\newcommand{\usp}{\>}

% Grouping
\let\parens\undefined
\newcommand{\parens}[1]{\text{(}#1\text{)}}

\let\brackets\undefined
\newcommand{\brackets}[1]{\text{[}#1\text{]}}

\let\braces\undefined
\newcommand{\braces}[1]{\text{\{}#1\text{\}}}


% Language

\let\regex\undefined
\newcommand{\regex}{\ensuremath{E}}

\let\reglang\undefined
\newcommand{\reglang}{\ensuremath{L_{R}}}

\let\lang\undefined
\newcommand{\lang}{\ensuremath{L}}

\let\alphabet\undefined
\newcommand{\alphabet}{\ensuremath{\Sigma}}

\let\empstr\undefined
\newcommand{\empstr}{\ensuremath{\varepsilon}}

\let\langmin\undefined
\newcommand{\langmin}{\ensuremath{\eta}}


% Measures

\let\premucount\undefined
\newcommand{\premucount}{\ensuremath{\mu_{0, c}}}






% Author
\title{Kernel Methods and Automata}
\author{Anton Xue}
% \date{\today}
\date{}

% Document
\begin{document}
\maketitle

\section{Introduction}
In this sketch we study the relation between graph kernels and
finite state machines.
We study previous work in algebraic formulation for automata theory
and kernel methods, with an interest in how these techniques
may be extended to formal methods.



\section{Preliminaries}


\subsection{Algebraic Automata Theory}
Previous work in formulation in algebraic foundations for automata
theory exist in literature~\cite{kuich2012semirings},
and much of our notation is taken from~\cite{cortes2004rational}.
We overload notation for addition (\(+\)) and multiplication (\(\cdot\))
in algebraic structures when possible to avoid clutter.
Similarly, the additive identity (\(0\)) and multiplicative identity (\(1\))
are also overloaded when possible.

\begin{definition}[Monoid]
  A monoid is an algebraic structure \(\parens{\mbb{K}, \cdot, 1}\) where:
  
  \begin{itemize}
    \item
      \(\mbb{K}\) is closed under monoid multiplication
      \(\type{\cdot}{\mbb{K} \times \mbb{K}}{\mbb{K}}\).

    \item
      \(1\) is the multiplicative identity.

  \end{itemize}

  When possible, we elide the \(\cdot\) in monoid multiplication to write
  \(ab\) instead of \(a \cdot b\).
  When multiplication \(\cdot\) is commutative,
  the system is known as a commutative monoid.
\end{definition}


\begin{definition}[Semiring]
  A semiring \(\parens{\mbb{K}, +, \cdot, 0, 1}\) is a system where:

  \begin{itemize}
    \item
      Semiring addition is a commutative monoid
      \(\parens{\mbb{K}, +, 0}\).

    \item
      Semiring multiplication
      \(\parens{\mbb{K}, \cdot, 1}\) is a monoid.

    \item
      \(0\) annihilates semiring multiplication.
  \end{itemize}

\end{definition}

A weighted finite state transducer (WFST) is a very
general transition system defined
using a semiring to specify transition weights.


\begin{definition}[Weighted Finite State Transducer]
  A weighted finite state transducer over a semiring \(\mbb{K}\) is
  a system
  \(\parens{\Sigma_I, \Sigma_O, Q, I, F, \Delta, \lambda, \rho}\) where:

  \begin{itemize}
    \item
      \(\Sigma_I\) is a finite input alphabet.

    \item
      \(\Sigma_O\) is a finite output alphabet.

    \item
      \(Q\) is a finite set of states.

    \item
      \(I \subseteq Q\) is the set of starting states.

    \item
      \(F \subseteq Q\) is the set of final states.

    \item
      \(\Delta \subseteq
          Q \times
          \parens{\Sigma_I \cup \braces{\varepsilon}} \times
          \mbb{K} \times
          \parens{\Sigma_O \cup \braces{\varepsilon}} \times
          Q \)
      is the transition function weighted by \(\mbb{K}\).
        
    \item
      \(\type{\lambda}{I}{\mbb{K}}\) is the initial state weight function.

    \item
      \(\type{\rho}{F}{\mbb{K}}\) is the final state weight function.

    \end{itemize}
\end{definition}

It should be noted that \(\Sigma_I\) and \(\Sigma_O\) can be
seen as the generators of the free monoids
\(\Sigma_I ^\star\) and \(\Sigma_O ^\star\),
which represents the set of all strings over
\(\Sigma_I\) and \(\Sigma_O\) respectively.

Although we are not yet interested in the full generality that
a WFST offers, it is still nice to see what is available to us
in terms of abstraction.

A special case of WFSTs that we are interested in
are non-deterministic finite automata,
whose specification in terms of WSFTs
is by Cortes~\cite{cortes2004rational}.
We introduce a simplified structure.

\begin{definition}[Non-deterministic Finite Automata]
  A non-deterministic finite automata is a system
  \(\parens{\Sigma, Q, \Delta, I, F}\) where:

  \begin{enumerate}
    \item
      \(\Sigma\) is a finite alphabet.

    \item
      \(Q\) is a finite set of states.

    \item
      \(\Delta \subseteq Q \times \Sigma \times Q\)
      is the transition function.

    \item
      \(I \subseteq Q\) is the set of initial states.

    \item
      \(F \subseteq Q\) is the set of final states.

  \end{enumerate}
\end{definition}

We do not permit \(\varepsilon\)-transitions in our definition,
which can be eliminated anyways~\cite{savage1998models}.



\subsection{Reproducing Kernel Hilbert Spaces}



\section{Automata Embeddings}



\section{Bounding Inner Products}



\printbibliography


\end{document}


