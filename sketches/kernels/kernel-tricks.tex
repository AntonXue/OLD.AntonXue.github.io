\documentclass[12pt]{article}

% Packages
\usepackage[margin=5em]{geometry} % 1 cm = 2.84528 em
\usepackage[backend=bibtex]{biblatex}
\usepackage{xcolor}
\bibliography{sources}
% \nocite{*}

\usepackage{lipsum}

% Paragraphs
\setlength{\parindent}{0em}
\setlength{\parskip}{1em}

% Includes
% Includes
\usepackage{amsfonts}
\usepackage{amsmath}
\usepackage{amssymb}
\usepackage{amsthm}
\usepackage{comment}
\usepackage[colorlinks]{hyperref}
\usepackage{letltxmacro}
\usepackage{listings}
\usepackage{stmaryrd}

% Spacing
\let\uspace\undefined
\newcommand{\uspace}{\ensuremath{\ }}

% Numbers
\let\N\undefined
\newcommand{\N}{\ensuremath{\mathbb{N}}}

\let\Z\undefined
\newcommand{\Z}{\ensuremath{\mathbb{Z}}}

\let\Zz\undefined
\newcommand{\Zz}{\ensuremath{\mathbb{Z}^{\geq 0}}}

\let\Zp\undefined
\newcommand{\Zp}{\ensuremath{\mathbb{Z}^{+}}}

\let\Q\undefined
\newcommand{\Q}{\ensuremath{\mathbb{Q}}}

\let\R\undefined
\newcommand{\R}{\ensuremath{\mathbb{R}}}

\let\Rz\undefined
\newcommand{\Rz}{\ensuremath{\mathbb{R}^{\geq 0}}}

\let\Rp\undefined
\newcommand{\Rp}{\ensuremath{\mathbb{R}^{+}}}

\let\C\undefined
\newcommand{\C}{\ensuremath{\mathbb{C}}}

\let\mcal\undefined
\newcommand{\mcal}[1]{\ensuremath{\mathcal{#1}}}

\let\mbb\undefined
\newcommand{\mbb}[1]{\ensuremath{\mathbb{#1}}}

% Grouping
\let\parens\undefined
\newcommand{\parens}[1]{\ensuremath{\left(#1\right)}}

\let\brackets\undefined
\newcommand{\brackets}[1]{\ensuremath{\left[#1\right]}}

\let\braces\undefined
\newcommand{\braces}[1]{\ensuremath{\left\{#1\right\}}}

\let\angles\undefined
\newcommand{\angles}[1]{\ensuremath{\left\langle#1\right\rangle}}

\let\ceil\undefined
\newcommand{\ceil}[1]{\ensuremath{\left\lceil#1\right\rceil}}

\let\floor\undefined
\newcommand{\floor}[1]{\ensuremath{\left\lfloor#1\right\rfloor}}

% Sets and Function Spaces
\let\type\undefined
\newcommand{\type}[3]{\ensuremath{#1 \colon #2 \to #3}}

\let\powset\undefined
\newcommand{\powset}[1]{\ensuremath{2^{#1}}}

\let\mod\undefined
\newcommand{\mod}{\ensuremath{\uspace\mathrm{mod}\uspace}}

\let\ker\undefined
\newcommand{\ker}{\ensuremath{\mathrm{ker}}}

\let\dom\undefined
\newcommand{\dom}{\ensuremath{\mathrm{dom}}}

\let\ran\undefined
\newcommand{\ran}{\ensuremath{\mathrm{ran}}}

\let\im\undefined
\newcommand{\im}{\ensuremath{\mathrm{im}}}

\let\coker\undefined
\newcommand{\coker}{\ensuremath{\mathrm{coker}}}

\let\codim\undefined
\newcommand{\codim}{\ensuremath{\mathrm{codim}}}

% Complex Numbers
\let\Re\undefined
\newcommand{\Re}{\ensuremath{\mathrm{Re}}}

\let\Im\undefined
\newcommand{\Im}{\ensuremath{\mathrm{Im}}}

% Probability
\let\Pr\undefined
\newcommand{\Pr}{\ensuremath{\mathrm{Pr}}}

\let\E\undefined
\newcommand{\E}{\ensuremath{\mathrm{E}}}

\let\Var\undefined
\newcommand{\Var}{\ensuremath{\mathrm{Var}}}

\let\Cov\undefined
\newcommand{\Cov}{\ensuremath{\mathrm{Cov}}}

% Integrals
\let\dee\undefined
\newcommand{\dee}[1]{\ensuremath{\uspace d #1}}

% Linear Algebra
\let\abs\undefined
\newcommand{\abs}[1]{\ensuremath{\left\lvert#1\right\rvert}}

\let\norm\undefined
\newcommand{\norm}[1]{\ensuremath{\left\lVert#1\right\rVert}}


% Computational Complexity
\let\class\undefined % Complexity class
\newcommand{\class}[1]{\ensuremath{\mathbf{#1}}}

\let\prob\undefined % Complexity problem
\newcommand{\prob}[1]{\ensuremath{\text{#1}}}

% Theorem environment
\theoremstyle{plain}
\newtheorem{theorem}{Theorem}

\theoremstyle{plain}
\newtheorem{lemma}{Lemma}

\theoremstyle{plain}
\newtheorem{claim}{Claim}

\theoremstyle{plain}
\newtheorem{fact}{Fact}

\theoremstyle{plain}
\newtheorem{remark}{Remark}

\theoremstyle{plain}
\newtheorem{definition}{Definition}

\theoremstyle{plain}
\newtheorem{example}{Example}

\theoremstyle{plain}
\newtheorem{question}{Question}

% Code environment
\lstdefinestyle{plainsty}{
  basicstyle=\small\ttfamily,
  language=C,
  xleftmargin=\parindent,
  aboveskip=1em,
  belowskip=1em,
  showspaces=false,
  showstringspaces=false,
  keywordstyle = {},
}

\lstnewenvironment{pcode}{\lstset{style=plainsty}}{}

\let\pinl\undefined
\newcommand{\pinl}{\lstinline[style=plainsty]}

\newcommand*{\SavedLstInline}{} % Allows plain code usage in math mode.
\LetLtxMacro\SavedLstInline\pinl
\DeclareRobustCommand*{\pinl}{%
  \ifmmode
    \let\SavedBGroup\bgroup
    \def\bgroup{%
      \let\bgroup\SavedBGroup
      \hbox\bgroup
    }%
  \fi
\SavedLstInline}

\let\ttcode\undefined
\newcommand{\ttcode}[1]{\small{\texttt{#1}}}


% Text markings

\let\tturl\undefined
\newcommand{\tturl}[1]{\href{#1}{\texttt{#1}}}

\let\red\undefined
\newcommand{\red}[1]{\textbf{\color{red}#1}}


% Abbreviations
\let\st\undefined
\newcommand{\st}{\ensuremath{\uspace\colon\uspace}}

\let\ow\undefined
\newcommand{\ow}{\ensuremath{\text{otherwise}}}


% \usepackage{stmaryrd}


\let\usp\undefined
\newcommand{\usp}{\>}

% Grouping
\let\parens\undefined
\newcommand{\parens}[1]{\text{(}#1\text{)}}

\let\brackets\undefined
\newcommand{\brackets}[1]{\text{[}#1\text{]}}

\let\braces\undefined
\newcommand{\braces}[1]{\text{\{}#1\text{\}}}


% Language

\let\regex\undefined
\newcommand{\regex}{\ensuremath{E}}

\let\regexes\undefined
\newcommand{\regexes}{\ensuremath{\mathcal{E}}}

\let\reglang\undefined
\newcommand{\reglang}{\ensuremath{L_{R}}}

\let\reglangs\undefined
\newcommand{\reglangs}{\ensuremath{\mathcal{L}_R}}

\let\lang\undefined
\newcommand{\lang}{\ensuremath{L}}

\let\langs\undefined
\newcommand{\langs}{\ensuremath{\mathcal{L}}}

\let\alphabet\undefined
\newcommand{\alphabet}{\ensuremath{\Sigma}}

\let\empstr\undefined
\newcommand{\empstr}{\ensuremath{\varepsilon}}

\let\langmin\undefined
\newcommand{\langmin}{\ensuremath{\eta}}


% Measures

\let\mucount\undefined
\newcommand{\mucount}{\ensuremath{\mu_{c}}}

\let\premucount\undefined
\newcommand{\premucount}{\ensuremath{\mu_{0, c}}}






% Author
\title{Kernel Methods and Automata}
\author{Anton Xue}
% \date{\today}
\date{}

% Document
\begin{document}
\maketitle

\section{Introduction}
In this sketch we study the relation between graph kernels and
finite state machines.
We study previous work in algebraic formulation for automata theory
and kernel methods, with an interest in how these techniques
may be extended to formal methods.



\section{Preliminaries}


\subsection{Algebraic Automata Theory}
Previous work in formulation in algebraic foundations for automata
theory exist in literature~\cite{kuich2012semirings},
and much of our notation is taken from~\cite{cortes2004rational}.
We overload notation for addition (\(+\)) and multiplication (\(\cdot\))
in algebraic structures when possible to avoid clutter.
Similarly, the additive identity (\(0\)) and multiplicative identity (\(1\))
are also overloaded when possible.

\begin{definition}[Monoid]
  A monoid is an algebraic structure \(\parens{\mbb{K}, \cdot, 1}\) where:
  
  \begin{itemize}
    \item
      \(\mbb{K}\) is closed under monoid multiplication
      \(\type{\cdot}{\mbb{K} \times \mbb{K}}{\mbb{K}}\).

    \item
      \(1\) is the multiplicative identity.

  \end{itemize}

  When possible, we elide the \(\cdot\) in monoid multiplication to write
  \(ab\) instead of \(a \cdot b\).
  When multiplication \(\cdot\) is commutative,
  the system is known as a commutative monoid.
\end{definition}


\begin{definition}[Semiring]
  A semiring \(\parens{\mbb{K}, +, \cdot, 0, 1}\) is a system where:

  \begin{itemize}
    \item
      Semiring addition is a commutative monoid
      \(\parens{\mbb{K}, +, 0}\).

    \item
      Semiring multiplication
      \(\parens{\mbb{K}, \cdot, 1}\) is a monoid.

    \item
      \(0\) annihilates semiring multiplication.
  \end{itemize}

\end{definition}


\begin{definition}[Weighted Finite State Transducer]
  A weighted finite state transducer over a semiring \(\mbb{K}\) is an
  \(8\)-tuple
  \(\parens{\Sigma_I, \Sigma_O, Q, I, F, \Delta, \lambda, \rho}\) where:

  \begin{itemize}
    \item
      \(\Sigma_I\) is a finite input alphabet.

    \item
      \(\Sigma_O\) is a finite output alphabet.

    \item
      \(Q\) is a finite set of states.

    \item
      \(I \subseteq Q\) is the set of starting states.

    \item
      \(F \subseteq Q\) is the set of final states.

    \item
      \(\Delta \subseteq
          Q \times
          \parens{\Sigma_I \cup \braces{\varepsilon}} \times
          \mbb{K} \times
          \parens{\Sigma_O \cup \braces{\varepsilon}} \times
          Q \)
      is the transition function weighted by \(\mbb{K}\).
        
    \item
      \(\type{\lambda}{I}{\mbb{K}}\) is the initial state weight function.

    \item
      \(\type{\rho}{F}{\mbb{K}}\) is the final state weight function.

    \end{itemize}
\end{definition}


\subsection{Reproducing Kernel Hilbert Spaces}



\section{Automata Embeddings}



\section{Bounding Inner Products}



\printbibliography


\end{document}


