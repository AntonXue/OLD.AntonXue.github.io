% LaTeX file for resume 
% This file uses the resume document class (res.cls)

\documentclass[margin]{res} 

\usepackage[pdftex]{hyperref}

% the margin option causes section titles to appear to the left of body text 
\textwidth=5.4in % increase textwidth to get smaller right margin
%\usepackage{helvetica} % uses helvetica postscript font
% (download helvetica.sty)
%\usepackage{newcent}   % uses new century schoolbook postscript font 

\let\linkurl\undefined
\newcommand{\linkurl}[1]{\href{#1}{\texttt{#1}}}

\begin{document} 

\name{Anton Xue\\[12pt]} % the \\[12pt] adds a blank line after name
 
\address{\textbf{Address} \\
         Levine Hall, 3330 Walnut St. \\
         Philadelphia, PA 19104}
 
\address{\textbf{Contact} \\
         \linkurl{antonxue@seas.upenn.edu} \\
         \linkurl{antonxue.github.io}}

\begin{resume} 
 
\section{Interests} 
Dynamical systems, programming languages, formal methods \\
Mathematical analysis, linear algebra, combinatorics

\section{Education} 
\textit{Ph.D. Computer and Information Science}
  \hfill 08/2019 -- Present \\
University of Pennsylvania

\textit{B.S. Mathematics (Intensive) and Computer Science}
  \hfill 08/2015 -- 05/2019 \\
Yale University


\section{Work Experience}
\textit{Research Intern} \hfill 06/2019 -- 08/2019 \\
Nokia Bell Labs

\textit{Research Assistant} \hfill 09/2015 -- 05/2019 \\
Yale University Department of Computer Science

\textit{Research Intern} \hfill 05/2018 -- 08/2018 \\
Harvard John A. Paulson School of Engineering and Applied Sciences

\textit{Research Intern} \hfill 05/2017 -- 08/2017 \\
Max Planck Institute for Software Systems

\textit{Software Engineering Intern} \hfill 05/2014 -- 08/2015 \\
Harvard Medical School


\section{Awards and Honors}
University of Pennsylvania ENIAC Fellowship \hfill 08/2019

Yale Computer Science Award \hfill 05/2019

National Science Foundation Graduate Research Fellowship \hfill 04/2019

Yale College Freshman Summer Research Fellowship \hfill 04/2016


\section{Conference Publications}
\textit{Data-Driven System Level Synthesis} \hfill 12/2020 \\
(in submission, arXiv: \url{https://arxiv.org/abs/2011.10674})

\textit{A Self-Certifying Compilation Framework for WebAssembly} \hfill 01/2021 \\
VMCAI 2021

\textit{Lazy Counterfactual Symbolic Execution} \hfill 06/2019 \\
PLDI 2019

\section{Workshop Publications}

\textit{G2Q: Haskell Constraint Solving} \hfill 08/2019 \\
Haskell Symposium 2019


\section{Presentations}
\textit{Towards a Self-Certifying Compiler for WebAssembly} \hfill 12/2019 \\
  IBM Programming Language Day 2019

\textit{Towards a Self-Certifying Compiler for WebAssembly} \hfill 10/2019 \\
  FMCAD 2019 Student Forum

\textit{Towards the Formalization and Analysis of R} \hfill 11/2018 \\
  FMCAD 2018 Student Forum
  % \hfill Nov 2018 \\
  % Formal Methods in Computer-Aided Design Student Forum, 11/2018

\textit{Building a Symbolic Execution Engine for Haskell} \hfill 11/2017 \\
  FMCAD 2017 Student Forum
  % \hfill Nov 2017 \\
  % Formal Methods in Computer-Aided Design Student Forum, 11/2017

\textit{Building a Symbolic Execution Engine for Haskell} \hfill 08/2017 \\
  TAPAS 2017
  % \hfill Aug 2017 \\
  % Tools for Automatic Program Analysis, 08/2017

\textit{A Symbolic Execution Framework for Haskell} \hfill 01/2017 \\
  POPL 2017 Student Research Competition
  % \hfill Jan 2017 \\
  % Principles of Programming Languages Student Research Competition, 01/2017


\section{Teaching}
\textit{Teaching Assistant}
  \hfill 05/2020 -- 12/2020 \\
    {CIS 515 Fundamentals of Linear Algebra and Optimization},
      Fall/2020 \\
    {CIS 160 Mathematical Foundations of Computer Science},
      Summer/2020 \\
University of Pennsylvania

\textit{Teaching Assistant}
  \hfill 09/2016 -- 05/2019 \\
 % \-\hspace{1em}
    {MATH 305 Real Analysis (Course Grader)},
      Spring/2019 \\
 % \-\hspace{1em}
    {CPSC 202 Mathematical Tools for Computer Science},
      Fall/2016, Fall/2017, Fall/2018 \\
 % \-\hspace{1em}
    {CPSC 366 Intensive Algorithms},
      Spring/2018 \\
 % \-\hspace{1em}
    {CPSC 365 Design and Analysis of Algorithms},
      Spring/2017 \\
Yale University

\section{Community}
\textit{Artifact Evaluation Committee} \hfill 03/2020 \\
PLDI 2020

\textit{Head Student Volunteer} \hfill 07/2019 \\
CAV 2019

\textit{Student Volunteer} \hfill 06/2019 \\
PLDI 2019

\textit{Department Student Advisory Committee} \hfill 08/2017 -- 05/2018 \\
Yale University Computer Science Department

\textit{Student Volunteer} \hfill 07/2017 \\
CAV 2017


\section{Software}
\textit{Self-Certified Optimizer for WebAssembly} \\
\linkurl{https://github.com/nokia/web-assembly-self-certifying-compilation-framework}\

\textit{G2 Symbolic Execution Engine for Haskell} \\
\linkurl{https://github.com/BillHallahan/G2}\

\textit{Simple-R Symbolic Execution Engine for R} \\
\linkurl{https://github.com/AntonXue/simple-r}

\textit{Multi-Terminal Interval Decision Diagrams} \\
\linkurl{https://github.com/dzufferey/mtidd}

\section{Technical}
\textit{Programming Languages} \\
Haskell, C, Python, Java, R, Scala, C\texttt{++}, SMTLIB, \LaTeX{}
 



\if false
% Tabulate Computer Skills; p{3in} defines paragraph 3 inches wide
\section{Computer \\ Skills}
   \begin{tabular}{l p{3in}}
    \underline{Languages:} & COBOL, Pascal,C, APL, Basic \\

     \underline{Software:} &  SPIRES, dBase III, Datastar database 
                        systems, GPSS simulation, FCS-EPS financial 
                        planning, SAS statistical analysis, 
                        Lotus1-2-3, MPSX optimization modeling 
   \end{tabular}
 \fi

\end{resume} 
\end{document} 


